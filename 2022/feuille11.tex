%\documentclass[reponses, utf8, 11pt]{feuille}
\documentclass[utf8, 11pt]{feuille}

\newcommand{\titredutd}{\textbf{TD11 --- Exemple numérique}}

\begin{document}


\begin{tcolorbox}[
        colback=gray!20,
        colframe=gray!20,
        width=\dimexpr\textwidth\relax, 
        arc=0pt,outer arc=0pt,
        ]

\texttt{Seules les calculatrices non communicantes et les notes manuscrites personnelles sont autorisées.}

\texttt{Les exercices sont totalement indépendants.}

\texttt{On notera $k_B$ la constante de Boltzmann et $h$ la constante de Planck.}

\end{tcolorbox}



% ______________________________________________________________________________
\section{\medium~Statistiques classique et quantiques : un exemple numérique}

On considère un système quantique dont l'Hamiltonien exhibe trois états propres non dégénérés d'énergies distinctes, respectivement 0, $\epsilon$ et $2\epsilon$, en équilibre avec un thermostat à la température $T$.

\medskip

\question
Le système est composé d'une particule sans spin. Représenter graphiquement les différents micro-états possibles du système. Calculer la fonction de partition $Z_1(\beta)$ de cette particule et son énergie moyenne $\langle E_1 (\beta) \rangle$.

\question
Le système est composé de deux particules sans spin \textit{discernables}. Représenter graphiquement les différents micro-états possibles du système. Calculer la fonction de partition $Z_{2d}(\beta)$ de ces deux particules et montrer que  $Z_{2d}(\beta)=Z_1(\beta)^2$. En déduire l'énergie moyenne $\langle E_{2d}(\beta) \rangle$. Tracer avec Python $\langle E_{2d}(\beta) \rangle$ en fonction du paramètre $x=\frac{k_BT}{\epsilon}$. 

\question
Le système est composé de deux bosons \textit{indiscernables}. Représenter graphiquement les différents micro-états possibles du système. Calculer la fonction de partition $Z_{2B}(\beta)$ de ces deux particules. Vérifier que  $Z_{2B}(\beta)=\frac{1}{2} [Z_1(\beta)^2+Z_1(2\beta)]$. En déduire l'énergie moyenne $\langle E_{2B}(\beta) \rangle$. Tracer avec Python $\langle E_{2B}(\beta) \rangle$ en fonction de $x$. 

\question
Le système est composé de deux fermions \textit{indiscernables} de spin nul (en totale violation du théorème spin-statistique !). Représenter graphiquement les différents micro-états possibles du système. Calculer la fonction de partition $Z_{2F}(\beta)$ de ces deux particules. Vérifier que  $Z_{2F}(\beta)=\frac{1}{2} [Z_1(\beta)^2-Z_1(2\beta)]$. En déduire l'énergie moyenne $\langle E_{2F}(\beta) \rangle$. Tracer avec Python $\langle E_{2F}(\beta) \rangle$ en fonction de $x$. 

\question
Qu'advient-il si les fermions ont un spin $\frac{1}{2}$ ? Reprendre la question précédente. Expliquer pourquoi cela revient à considérer que tous les niveaux d'énergie sont dégénérés deux fois.

\medskip

On considère désormais qu'outre la température, on fixe le potentiel chimique $\mu$.

\medskip

\question
Rappeler les expressions des taux d'occupation de chaque état propre en fonction de son énergie, de $T$ et de $\mu$ selon que les particules sont des bosons ou des fermions.

\question
Donner les expressions correspondantes (pour notre système avec trois états propres) de l'énergie moyenne $\langle E (\beta,\mu) \rangle$ et du nombre moyen de particules $\langle N(\beta,\mu) \rangle$.

\question
On veut un nombre moyen de bosons égal à deux. Calculer numériquement la fugacité $f=\exp(\beta \mu)$ pour quelques valeurs du paramètre $x$, puis l'énergie moyenne correspondante. Comparer avec les résultats de la question 3.  



% ______________________________________________________________________________
\section{\medium~Modèle à  deux niveaux d'un semi-conducteur intrinsèque}

On se propose d'étudier la conductivité électrique d'un semi-conducteur intrinsèque, c'est-à -dire ne contenant pas d'impuretés. On considère donc un modèle dans lequel les bandes de valence et de conduction sont assimilées à  deux niveaux d'énergie $\epsilon_1$ et $\epsilon_2$, de même dégénérescence $g=g_1=g_2$. On posera $\Delta \epsilon=\epsilon_2-\epsilon_1 > 0$. Soit $N$ le nombre d'électrons se trouvant dans la bande de valence (c'est-à -dire dans ce modèle sur $\epsilon_1$) au zéro absolu:  à  cette température, la bande de conduction est vide. On suppose donc que $N=g$. Soient $T$ la température du système et $\mu$ son potentiel chimique.

\question{A quelle statistique obéissent les électrons ? En déduire le nombre d'électrons $N_1$ et $N_2$ se trouvant respectivement sur les niveaux $\epsilon_1$ et $\epsilon_2$ à  l'équilibre thermique. Quelle relation lie $N_1$, $N_2$ et $N$ ? }

\question{On pose $f=\exp(\beta \mu)$, $f_1=\exp(\beta \epsilon_1)$ et $f_2=\exp(\beta \epsilon_2)$.  Déduire de la relation précédente l'expression de $f$ en fonction de $f_1$ et $f_2$. Montrer que, quelle que soit la température, $\mu=\frac{\epsilon_1+\epsilon_2}{2}$.}

\question{On note  $P_1$ et $P_2$, le nombre de trous ou places vides dans chacun des niveaux 1 et 2. Exprimer $N_1$ et $N_2$, ainsi que $P_1$ et $P_2$ en fonction de $g, \Delta \epsilon$ et $k_B T$. Quelles sont les limites de ces quantités aux basses et aux hautes températures ?}

\question{Comparer ces résultats avec ceux que l'on obtient lorsqu'on applique la statistique classique de Maxwell-Boltzmann à  ce système.}

\question{Donner les expressions de l'énergie moyenne $\bar{E}$ et de la capacité calorifique $C$ en fonction de la température.}

\question{On admet que $C$ présente un maximum lorsque $2\frac{k_BT}{\Delta \epsilon}=0,42$. Sachant que $\Delta \epsilon=1$ eV, calculer la température correspondante. En déduire que dans les semi-conducteurs usuels, seule la partie croissante de la capacité calorifique avec la température peut être observée.}

\question{Toujours dans le cas $\Delta \epsilon=1$ eV, simplifier les expressions obtenues de $N_2$ et $P_1$ lorsque la température est proche de l'ambiante (300 K).}

\question{La conductivité électrique est donnée par la formule

\begin{align*}
\sigma=e\left(\frac{N_2}{V} \mu_-+\frac{P_1}{V} \mu_+\right),
\end{align*}
où $e$ est la charge élémentaire, $\mu_\pm$ les mobilités respectives des électrons et des trous dans ce cristal, quantités que l'on supposera indépendantes de la température. Pour le silicium, $n=\frac{N}{V}= 3,1.10^{19}$ cm$^{-3}$, $\Delta \epsilon=1,1$ eV, $\mu_+=400$ cm$^2$.(V.s)$^{-1}$ et $\mu_-=1600$ cm$^2$.(V.s)$^{-1}$. Calculer la conductivité électrique du silicium pur à  $T=300$ K et $T=1000$ K. Tracer la courbe théorique de la conductibilité en fonction de la température dans une représentation où le graphe de la fonction est linéaire. Expliquer comment la mesure de $\sigma$ permet de déterminer $\Delta \epsilon$. }




\end{document}
