\documentclass[a4paper, landscape, utf8, 10pt]{article}

\usepackage{fiche}
\usepackage[landscape]{geometry}

\begin{document}
\begin{multicols}{2}

\nom{46}{40}

\bigskip

%--------------------------------------------------------------------------------------
\exo{1}{Système à quatre niveaux}{20}

\sol{1}{1}{\À basse temp. $k_BT \ll \epsilon$, seul le fondamental est peuplé: $E=-3N \epsilon$}{1} \\
\sol{1}{1}{\À haute temp. $k_BT \gg \epsilon$, équirépartition entre les niveaux: $E \simeq 0$}{1} \\
\sol{1}{2}{$z(\beta)=e^{3\beta \epsilon}+3e^{\beta \epsilon}+3e^{-\beta \epsilon}+e^{-3\beta \epsilon}=(e^{\beta \epsilon}+e^{-\beta \epsilon})^3 =8 \cosh^3{\beta \epsilon}$}{2} \\
\sol{1}{3}{$u=-\frac{\partial \ln z}{\partial \beta}=-3 \frac{\partial \ln \cosh{\beta \epsilon}}{\partial \beta}=-3 \epsilon \frac{\sinh{\beta \epsilon}}{\cosh{\beta \epsilon}}= -3 \epsilon \tanh{\beta \epsilon} $}{2} \\
\sol{1}{4}{$c=\frac{\partial u}{\partial T}=-k_B \beta^2 \frac{\partial u}{\partial \beta}=3 k_B \frac{(\beta \epsilon)^2}{\cosh^2{\beta \epsilon}}$}{2} \\
\sol{1}{4}{En fonction de $\beta$, $\nearrow$ puis $\searrow$  - anomalie de Schottky}{1} \\
\sol{1}{5}{$-3N \epsilon \le E \le +3N \epsilon$ }{1} \\
\sol{1}{6}{$N=N_{+3}+N_{+1}+N_{-1}+N_{-3}$ et $\frac{E}{\epsilon}=3(N_{+3}-N_{-3})+N_{+1}-N_{-1}$}{2} \\
\sol{1}{7}{$\Omega(N_{+3},N_{+1},N_{-1},N_{-3})=\binom{N_{+3}}{N} \binom{N_{+1}}{N-N_{+3}}3^{N_{+1}}\binom{N_{-1}}{N-N_{+3}-N_{+1}}3^{N_{-1}}$}{1} \\
\sol{1}{7}{$\Omega(N_{+3},N_{+1},N_{-1},N_{-3})=3^{N_{+1}+N_{-1}}\frac{N!}{N_{+3}! N_{+1}! N_{-1}! N_{-3}!}$}{1} \\
\sol{1}{8}{$\Omega(E,N)=\sum\limits_{N_{+3},N_{+1},N_{-1},N_{-3}} \Omega(N_{+3},N_{+1},N_{-1},N_{-3}) \times\\ \delta_{N,N_{+3}+N_{+1}+N_{-1}+N_{-3}} \times \delta_{\frac{E}{\epsilon},3(N_{+3}-N_{-3})+N_{+1}-N_{-1}}$}{2} \\
\sol{1}{9}{Cherc. le term max car $\ln \Omega(E,N) \simeq \ln \Omega_M(N_{+3},N_{+1},N_{-1},N_{-3})$ }{1} \\
\sol{1}{9}{\'Ecrire $\Omega(N_{+3},N_{+1},N_{-1},N_{-3})$ sous forme $\exp$ et utiliser Stirling }{1} \\
\sol{1}{9}{Utiliser deux mult. de Lagrange $(\lambda, \mu)$ pour  les deux contraintes }{1} \\
\sol{1}{9}{Cherc. l'extremum de l'$\exp$ / variables $N_{+3},N_{+1},N_{-1},N_{-3}, \lambda, \mu$  }{1} \\



%--------------------------------------------------------------------------------------
\exo{2}{Au-delà de l'harmonique}{26}

\sol{2}{1}{Deux degrés de liberté quadratique énergétiques : équipartition. }{1} \\
\sol{2}{1}{\textbf{ou} calc. intégral $z(\beta)=\iint_{-\infty}^{+\infty} \frac{dpdx}{h} \exp{-\beta(\frac{p^2}{2m}+\frac{Kx^2}{2})}=\frac{2\pi}{h\beta} \sqrt{\frac{m}{K}}$. }{1} \\
\sol{2}{1}{$u =-\frac{\partial \ln z}{\partial \beta} = \frac{k_B T}{2}+\frac{k_B T}{2}=k_B T$}{1} \\
\sol{2}{1}{$c = \frac{\partial u}{\partial T}= k_B$}{1} \\
\sol{2}{2}{$P(x) \propto \e^{- \beta \frac{ K x^2}{2}}$ et $\int_{-\infty}^{+\infty} dx P(x) =1$.}{1} \\
\sol{2}{2}{$P(x) = \left( \frac{\beta K}{2 \pi} \right) ^{1/2} \e ^{- \beta \frac{ K x^2}{2}}$}{1} \\
\sol{2}{3}{$\langle x \rangle = \int_{-\infty}^{+\infty} x P(x) \D x = 0$ (fonction impaire, support symétrique)}{1} \\
\sol{2}{3}{$\langle x^2 \rangle = \int_{-\infty}^{+\infty} x^2 P(x) \D x = \left( \frac{\beta K}{2 \pi} \right) ^{1/2} \int_{-\infty}^{+\infty} x^2 \e ^{- \beta \frac{ K x^2}{2}}  \D x =\frac{1}{\beta K}$}{1} \\
\sol{2}{4}{$\delta E(x) = \frac{ \delta K x^2}{2}= E(x) \frac{\delta K}{K}$}{1} \\
\sol{2}{4}{$\delta P(x) = \left( \frac{\beta K}{2 \pi} \right) ^{1/2}[\frac{\delta K}{2K}-  \beta \frac{\delta K x^2}{2} ]\e ^{- \beta \frac{ K x^2}{2}}= P(x) \frac{\delta K}{K} \left( \frac{1}{2} - \beta E(x) \right)$}{1} \\
\sol{2}{5}{$\delta T = \frac{\delta K}{K} \int_{-\infty}^{+\infty} P(x) E(x) dx= \frac{\delta K}{K} \frac{k_B T}{2}$. Travail}{2} \\
\sol{2}{6}{$\delta C = \frac{\delta K}{K} \int_{-\infty}^{+\infty} E(x) P(x) \left( \frac{1}{2} - \beta E(x) \right) = \frac{\delta K}{K} \left( \frac{k_BT}{4}-\beta \frac{3}{4\beta^2} \right) $. }{1} \\
\sol{2}{6}{$\delta C = - \frac{\delta K}{K} \frac{k_B T}{2}$. Chaleur}{1} \\
\sol{2}{7}{On constate que $\delta C =-\delta T$ ce qui est normal car}{0} \\
\sol{2}{7}{$dU=\delta C+\delta T \texttt{ (1er principe)} = 0 \texttt{ (Transf. isotherme)}$.}{2} \\
\sol{2}{8}{$W=\int_K^{K'} \delta T=\frac{k_B T}{2} \int_K^{K'} \frac{\delta K}{K}= \frac{k_B T}{2} \ln (\frac{K'}{K}).  $}{1} \\
\sol{2}{9}{Transf. réversible isotherme: $\Delta S =\frac{Q}{T}=-\frac{W}{T}=- \frac{k_B}{2} \ln (\frac{K'}{K})$ }{2} \\
\sol{2}{10}{$Z =  \int \e ^{- \frac{\beta p^2}{2m}} \frac{\D p}{h} \int \e ^{- \beta V(x)} dx$; $\e^{- \beta V(x) } \simeq \e ^{-  \frac{ \beta K x^2}{2}} (1+\beta gx^3+ \beta fx^4)$}{1} \\
\sol{2}{10}{$Z =  \frac{2\pi}{h\beta} \sqrt{\frac{m}{K}}(1+\beta g \langle x^3 \rangle+ \beta f \langle x^4 \rangle + \ldots)= \frac{2\pi}{h\beta} \sqrt{\frac{m}{K}}(1+f \frac{3}{\beta K^2} + \ldots )$}{1} \\
\sol{2}{11}{$u^{\prime} = - \partial_{\beta} \ln Z = k_B T \left( 1+\frac{3f}{K^2} k_B T \right)$; $c^{\prime} = \partial_{T} u^{\prime} = k_B+\frac{6f}{K^2} k_B^2 T$}{2} \\
\sol{2}{11}{Plus de constance de $c$ mais cela ne dépend pas de $g$.}{1} \\
\sol{2}{12}{$\langle x \rangle =\frac{\int_{-\infty}^{+\infty} x \e ^{- \beta V(x)} \D x}{\int_{-\infty}^{+\infty} \e ^{- \beta V(x)} \D x} \simeq \beta g \frac{\int_{-\infty}^{+\infty} x^4 \e ^{- \beta \frac{ K x^2}{2}} \D x}{\int_{-\infty}^{+\infty} \e ^{- \beta \frac{ K x^2}{2}} \D x} = \frac{3gk_B}{K^2}T$}{2} \\
\sol{2}{13}{Terme en $g$ linéaire en $T$. L'oscillateur ne vibre plus autour de sa position d'équilibre et l'écart varie avec $T$: dilatation. }{1} \\



\end{multicols}
\end{document}