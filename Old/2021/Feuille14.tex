%\documentclass[reponses, utf8, 11pt]{feuille}
\documentclass[utf8, 11pt]{feuille}

\newcommand{\titredutd}{\textbf{TD14 --- Gaz de Bosons}}

\begin{document}


\begin{tcolorbox}[
        colback=gray!20,
        colframe=gray!20,
        width=\dimexpr\textwidth\relax, 
        arc=0pt,outer arc=0pt,
        ]

\texttt{Seules les calculatrices non communicantes et les notes manuscrites personnelles sont autorisées.}

\texttt{Les exercices sont totalement indépendants.}

\texttt{On notera $k_B$ la constante de Boltzmann et $h$ la constante de Planck.}

\end{tcolorbox}



% ______________________________________________________________________________
\section{\medium~Gaz de bosons à deux dimensions}

On étudie un gaz parfait constitué de $N$ bosons indépendants de masse $m$ et de spin nul, libres mais astreints à se déplacer sur une surface d'aire $S$ et à l'équilibre à la température $T$.

\question
Déterminer la densité d'état en énergie $\rho_{2D}(\epsilon)$ (en appliquant des conditions aux limites périodiques pour le vecteur d'onde). 

\question
Rappeler quel est le nombre moyen d'occupation $n(\epsilon)$ d'un état quantique d'énergie $\epsilon$ lorsque le potentiel chimique du gaz est égal à $\mu$. Pourquoi a-t-on $\mu<0$ ?

\question
En déduire la relation implicite qui détermine le potentiel chimique $\mu$ en fonction de $N$, $S$ et $T$. Dans toute la suite, on notera $f=e^{\beta \mu}$

\question
Montrer que $\frac{N}{S}\Lambda^2=-\ln (1-f)$ où $\Lambda$ est une longueur que l'on exprimera en fonction des données.

\question
Vérifier que la formule précédente vous redonne bien, dans la limite classique (non quantique)
l'expression de la fugacité d'un gaz parfait classique.

\question
Conclure quant à l'existence, ou l'absence d'un phénomène de condensation de Bose-Einstein en deux dimensions.

\medskip

On piège à présent notre assemblée d'atomes par un potentiel harmonique magnétique $V(r)=\frac{1}{2}m\omega^2 r^2$. Les états propres de chaque atome sont indexés par deux entiers $i, j \geq 0$, et leur énergie associée est $\epsilon_{i,j} = \hbar \omega (i + j)$. 

\question
Dans le contexte expérimental qui nous concerne (des atomes de rubidium 87), on a  $T \sim 100$ nK et $\omega \sim 2\pi \times 10$ Hz. Estimer numériquement $\beta \hbar \omega$.

\question
Exprimer, en fonction de $n$, le nombre $g_n$ d'états accessibles à un atome
occupant le niveau d'énergie $n \hbar \omega$.

\question
Exprimer $N_0$ le nombre d'atomes occupant l'état fondamental, en fonction de $f$.

\question
Exprimer $N_e$ le nombre d'atomes occupant des états excités sous la forme d'une somme sur des entiers $\geq 1$, d'une fonction de $\frac{N}{S}, \beta \hbar \omega$ et $f$. 

\question
Compte-tenu de l'application numérique ci-dessous, on admet que l'on peut approcher $N_e$ par une intégrale. On introduit la fonction
$$
g_2(f)=\frac{1}{\Gamma(2)} \int_0^{+\infty} dx \frac{x}{\frac{e^x}{f}-1} = \sum_{k\geq 1} \frac{f^k}{k^2}
$$
Exprimer $N_e$ en fonction de $g_2(f)$ et des données.

\question
$g_2$ est une fonction croissante. Quel est le maximum possible pour $N_e$ en fonction de la température ?  On donne $\sum_1^{+\infty} \frac{1}{k^2}=\frac{\pi^2}{6}$.

\question
Conclure quant à l'existence, ou l'absence, d'un phénomène de condensation de Bose-Einstein en deux dimensions en présence d'un piège harmonique.



% ______________________________________________________________________________
\section{\hard~Faux thons}
Une cavité de volume $V$ et à l'équilibre à la température $T$  est le siège d'ondes électromagnétiques. On admet que ces ondes peuvent être décrites par une assemblée de photons obéissant à la statistique de Bose-Einstein à potentiel chimique nul. 

\`A une onde de fréquence $\nu$, on associe des photons d'énergie $\epsilon=h \nu= cp$ où $c$ est la vitesse de la lumière et $p$ l'impulsion (la quantité de mouvement) du photon.

\question
On admet que la densité d'états dans l'espace des impulsions est $g(p) d^3p =2 \frac{V}{h^3} 4 \pi p^2 dp$. (le facteur 2 est dû au fait que pour une impulsion donnée, un photon peut avoir deux états possibles de polarisation). Calculer la densité d'états en fonction de l'énergie.

\question
Quel est le nombre moyen   $n(\epsilon)$ de photons dans un mode d'énergie $\epsilon$ ?

\question
On désigne par $\rho(\nu, T)$ l'énergie par unité de volume des photons ayant leur fréquence comprise entre $\nu$ et $\nu+d\nu$ (appelée aussi densité spectrale d'énergie). Déterminer $\rho(\nu, T)$.

\question
La densité spectrale d'énergie est mesurable expérimentalement. Quelle est la forme de $\rho(\nu, T)$ aux basses fréquences (loi de Rayleigh-Jeans) ?  aux hautes fréquences (loi de Wien) ? Montrer que la loi de Rayleigh-Jeans peut se retrouver par le théorème d'équipartition de l'énergie. On admet que $\rho(\nu, T)$ présente entre ces deux régimes un maximum pour une fréquence $\nu_M$ qui vérifie $h \nu_M=2,82 k_B T$.

\question
Déduire de $\rho(\nu, T)$ l'expression de la densité d'énergie $u(T)$ dans l'enceinte. On donne
$$
\int_0^{+\infty} \frac{x^3 \ dx}{e^x-1}=\frac{\pi^4}{15}
$$

\medskip

On perce un petit trou de surface $A$ dans cette cavité d'où s'échappe une partie du rayonnement. On définit le pouvoir émissif $W$ de cette cavité comme étant le flux d'énergie sortant de l'orifice (par unité de temps et de surface).

\question
Calculer le nombre $d^3 n_{\epsilon, \Omega}$ de photons par unité de volume ayant leur énergie comprise entre $\epsilon$ et $\epsilon+ d\epsilon$ et leur impulsion pointant dans un angle solide $d\Omega$.

\question
En déduire le nombre de photons ayant leur énergie comprise entre $\epsilon$ et $\epsilon+ d\epsilon$ et leur impulsion pointant dans un angle solide $d\Omega$ qui quittent la cavité pendant le temps $dt$. 

\question
Calculer $W$ en fonction de $u(T)$.  Montrer que l'on a $W=\sigma T^4$ et calculer $\sigma$.


\end{document}
