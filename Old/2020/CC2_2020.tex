\documentclass[utf8, 11pt]{feuille}

\newcommand{\titredutd}{\textbf{CC2 --- 2020}}

\begin{document}

\begin{center}
    \Large {\bf Contrôle continu}
    
    Lundi 26 octobre 2020 - durée: 1h30
\end{center}

Les calculatrices non programmables sont permises mais les documents sont strictement interdits.
Les téléphones portables doivent être éteints et rangés. Les trois exercices sont largement indépendants. Les questions taguées avec un pique sont un peu plus délicates.

\medskip

On rappelle les valeurs du nombre d'Avogadro ${\cal N_A}=6,02 \times 10^{23}$, de la constante de Boltzmann $k_B=1,38 \times 10^{-23}$ J.K$^{-1}$ et de leur produit, la constante des gaz parfaits $R=8,315$ J.K$^{-1}$. 



% ______________________________________________________________________________
\section{Piégeage d'électrons sur des sites}

Un système isolé est constitué de $N$ électrons piégés sur les $N$ sites d'un solide cristallin. Chaque site peut accueillir au plus deux électrons et on a donc trois états possibles pour chacun des sites~:

\begin{itemize}
\item aucun électron n'est piégé: on prend cet état comme zéro d'énergie.
\item un électron est piégé: l'énergie de cet état est alors $-\epsilon<0$.
\item deux électrons sont piégés: l'énergie de cet état est $-2\epsilon+g$ où $g>0$. 
\end{itemize}

On note $n_0, n_1$ et $n_2$ le nombre de sites occupés respectivement par 0, 1 ou 2 électrons.

\question
Selon vous, pourquoi ne peut-on pas mettre plus de deux électrons par site et pourquoi $g>0$ ?

\question
Exprimer le nombre $N$ de sites du solide en fonction de $n_0, n_1$ et $n_2$. Faire de même pour le nombre $N$ d'électrons en fonction de $n_1$ et $n_2$. Montrer que $n_0=n_2$.

\question
Exprimer l'énergie $E$ en fonction de $\epsilon, g, n_1$ et $n_2$. Entre quelles limites peut-on choisir la valeur de $E$ ?

\question
Déterminer $n_0, n_1$ et $n_2$ en fonction de $E, N, \epsilon$ et $g$.

\question
Justifier que le nombre $\Omega(E,N)$ de micro-états d'énergie $E$ soit égal à  $\Omega(E,N)=\frac{N!}{n_0!n_1!n_2!}$.

\question{En déduire l'entropie $S(E,N)$ du système dans la limite thermodynamique (où tous les nombres sont de l'ordre du nombre d'Avogadro) en fonction de $N,n_1$ et $n_2$. On utilisera la formule de Stirling que l'on rappellera au préalable.}

\question
Calculer la température micro-canonique $T$ du système et montrer qu'elle peut s'exprimer sous la forme $\frac{1}{T}=\frac{2k_B}{g}\ln(\frac{n_1}{n_2})$.

\question
En déduire l'expression de $E$ en fonction de $N, T, \epsilon$ et $g$. Quelle est la limite de $E$ lorsque $k_B T \gg g$ ? Comment interprétez vous le résultat lorsque $g=0$ ?

\question
Calculer $n_0, n_1$ et $n_2$ en fonction de $N$,  $T, \epsilon$ et $g$. Commenter les limites de ces nombres lorsque $k_B T \ll g$ et lorsque $k_B T \gg g$.



% ______________________________________________________________________________
\section{La croissance vertigineuse du nombre de micro-états}

On considère un système isolé de $N$ particules d'énergie interne $E$. On admet que le nombre de micro-états $\Phi(E,N)$  d'énergie inférieure à $E$ est de la forme $\Phi(E,N)=A(N)E^{\alpha N}$ où $\alpha$ est un nombre proche de l'unité.

\question
Rappeler la nature de l'énergie interne d'un gaz parfait monoatomique comme l'argon. Justifier sommairement pourquoi, dans ce cas, $\alpha=\frac{3}{2}$.

Pour la suite, on prend pour les applications numériques $\alpha=\frac{3}{2}, N={\cal N}_A$, le nombre d'Avogadro et $E= 6$ kJ.

\question
Calculer le ratio $\frac{\Phi(E+\delta E,N)}{\Phi(E,N)}$ pour $\delta E= 6 $ nJ puis  $\delta E= 0,375 $ eV.

\question
Expliquer pourquoi, pour toute valeur \og raisonnable \fg \ de $\delta E$, le nombre $\Omega_{\delta E}(E,N)$ d'états d'énergie compris entre $E$ et $E+\delta E$ près est égal à $\Phi(E+\delta E,N)$. Expliquer l'affirmation que \og dans les espaces de dimensions très élevées, tout le volume est dans la surface \fg.

\question
En déduire que, dans la limite thermodynamique, l'entropie $S_{\delta E}(E,N)=k_B \ln \Omega_{\delta E}(E,N)$ ne dépend pas de ${\delta E}$ et qu'on a
$$
S(E,N) \simeq k_B \ln \Omega_{\delta E}(E,N)\simeq k_B \ln \Phi(E+\delta E,N) \simeq k_B \ln \Phi(E,N). \nonumber \\
$$



% ______________________________________________________________________________
\section{La distribution de Boltzmann dans l'ensemble micro-canonique}

On rappelle que $N$ atomes de masse $m$ d'un gaz parfait monoatomique isolé d'énergie $E$ dans un volume $V$, le nombre de micro-états $\Phi(E,V,N)$  d'énergie inférieure à $E$ est de la forme
$$
\Phi(E, V, N)= \frac{V^N}{N!}\frac{(\frac{2\pi m E}{h^2})^{\frac{3N}{2}}}{(\frac{3N}{2})!}
$$

\question
Calculer l'entropie $S(E,V,N)$ du gaz et vérifier qu'elle est extensive. Expliquer pourquoi l'extensivité nécessite de tenir compte de l'indiscernabilité des atomes.

\question
Calculer la température micro-canonique $T$ du système et exprimer $E$ en fonction de $N$ et de $T$. Quel résultat bien connu retrouve-t-on ?

On isole par la pensée une particule. 

\question
Montrer que la probabilité que cette particule soit située en $\vec r$ à $d\vec r$ près avec la quantité de mouvement $\vec p$ à $d\vec p$ près s'exprime comme
$$
P(\vec r,\vec p)d\vec r d\vec p=\frac{d\vec r d\vec p}{N h^3} e^{\frac{1}{k_B}\left[S(E-\frac{p^2}{2m},V,N-1)-S(E,V,N) \right]}
$$

\question
Développer $S(E-\frac{p^2}{2m},V,N-1)-S(E,V,N)$ au premier ordre et en déduire que la probabilité que la particule ait la quantité de mouvement $\vec p$ à $d\vec p$ est proportionnelle à $e^{-\frac{p^2}{2mk_BT}}$. Quel résultat bien connu retrouve-t-on ?


\end{document}


