%\documentclass[reponses, utf8, 11pt]{feuille}
\documentclass[utf8, 11pt]{feuille}

\newcommand{\titredutd}{\textbf{TD4 --- L'oscillateur harmonique quantique en micro-canonique}}

\begin{document}


\begin{tcolorbox}[
        colback=gray!20,
        colframe=gray!20,
        width=\dimexpr\textwidth\relax, 
        arc=0pt,outer arc=0pt,
        ]

\texttt{Seules les calculatrices non communicantes et les notes manuscrites personnelles sont autorisées.}

\texttt{Les exercices sont totalement indépendants.}

\texttt{On notera $k_B$ la constante de Boltzmann et $h$ la constante de Planck.}

\end{tcolorbox}



% ______________________________________________________________________________
\section{\medium~Système d'oscillateurs harmoniques quantiques}

Un système constitué de $N$ oscillateurs harmoniques {\it quantiques} de pulsation $\omega$ à une dimension, discernables et indépendants, est isolé, son énergie étant égale à $E$. Pour rappel, l'énergie de l'oscillateur $i=1,\dots,N$ est donnée par $e_i=(n_i+\frac{1}{2})\hbar \omega$, où $n_i$ est le nombre quantique d'excitation de l'oscillateur.

\question
Donner l'expression de l'énergie du système en fonction des nombres quantiques d'excitation $n_i$, où $i=1\dots N$.

\medskip
D'après ce qui précède fixer l'énergie $E$ revient à fixer la valeur de la somme des nombres quantiques $n_i$ à une valeur
$M$ :
$$
\sum_{i=1}^N n_i=M
$$
Soit $\varOmega(N,E)$ le nombre de micro-états du système de $N$ oscillateurs dont l'énergie vaut $E$.
 
\question
Commencer par calculer $\varOmega(N,E)$ pour $N$ quelconque et pour $M=0,1$ puis 2.

\question
Calculer à présent $\varOmega(N,E)$ pour $N=2$ et $M=3$.
	
\question
Calculer $\varOmega(N,E)$ dans le cas général.\\
Indice : ce problème est équivalent à trouver le nombre de façons de répartir $M$ objets dans $N$ boîtes distinctes.

\question
Vérifier l'expression en calculant par une sommation directe la dégénérescence des niveaux d'énergie d'un oscillateur harmonique tridimensionnel ($N=3$), c'est-à-dire le nombre de possibilités d'avoir $n_x+n_y+n_z=M$.

\question
Dans la limite des hautes énergies ($N$ étant fixé et $M \gg N$), montrer que l'on a:
$$
\varOmega(N,E)\simeq {1\over (N-1)!}\left({E\over \hbar  \omega}\right)^{N-1}
$$

\question
En déduire l'entropie micro-canonique dans la limite des hautes énergies, puis la température. Inverser cette relation pour obtenir $E(T)$.


% ______________________________________________________________________________
\section{\medium~L'OHQ dans tous ses (micro-)états}

On considère dans un premier temps une collection de $N $ oscillateurs harmoniques quantiques 1{\sc d} identiques, faiblement couplés entre eux et isolés du reste de l'univers. Le nombre de quanta d'énergie qu'ils se partagent est égal à $M$. Dans toute la suite, on considérera  que $N \gg 1$ et $M \gg 1$.

\question
Calculer l'énergie totale $E(N,M)$ et le nombre de micro-états $\Omega(N,M)$ du système, puis $\ln \Omega(N,M)$.

\question
Soit un {\sc ohq} particulier. Calculer le nombre de micro-états $\Omega(N,M|m)$ du système pour lequel cet {\sc ohq} possède exactement $m$ quanta d'énergie. En déduire la probabilité $p(m)$ qu'un {\sc ohq} contienne exactement $m$ quanta d'énergie.

\question
On introduit $\overline m= \frac{M}{N}$. Quelle est l'interprétation de $\overline m$ ? Montrer que dans l'hypothèse où $M \gg m$, $p(m)$ peut être approché par
$$
p(m) \sim \frac{1}{1+\overline m} \, \left(\frac{\overline m}{1+\overline m}\right)^m
$$
Tracer $p(m)$. Montrer que même dans l'approximation précédente $\sum_{m=0}^{m=M}p(m)=1$.

\medskip
On considère désormais deux collections d'{\sc ohq}, appelées 1 et 2,  comportant $N_1$ et $N_2$ {\sc ohq} respectivement et $M_{1i}$ (resp. $M_{2i}$) quanta d'énergie. Initialement isolées, elles sont mises en contact thermique à l'instant initial pour former une collection unique de $N=N_1+N_2$ {\sc ohq} faiblement couplés avec $M=M_{1i}+M_{2i}$ quantas. Tous ces nombres sont $\gg 1$.

\question
Calculer le nombre de micro-états du système réuni juste avant le contact $\Omega_i$ puis juste après $\Omega_f$. Comparer $\Omega_i$  et $\Omega_f$.

\question
Grâce au contact et au faible couplage, le nombre de quanta de 1 est désormais libre de fluctuer par échange d'énergie avec 2. Exprimer littéralement la probabilité $P(M_1)$ pour que la collection 1 possède $M_1$ quantas exactement.  

\question
En vous aidant de la formule de Stirling pour écrire $\ln P(M_1)$, calculer la valeur $\overline {M_1}$ de $M_1$ qui rend $P(M_1)$ maximum. Interpréter le résultat obtenu.

\question
Calculer le nombre $\Omega_e$ de micro-états du système correspondant à  $\overline{M_1}$ quanta dans 1. Comparer $\Omega_e$  à $\Omega_i$  et $\Omega_f$ , puis $\ln \Omega_e$  à $\ln \Omega_i$  et $\ln \Omega_f$. Que constate-t-on avec bonheur ?

\question
Expliquer (avec des mots) ce qu'il advient quand on met en contact thermique deux systèmes.



\end{document}
