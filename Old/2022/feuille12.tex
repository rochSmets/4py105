%\documentclass[reponses, utf8, 11pt]{feuille}
\documentclass[utf8, 11pt]{feuille}

\newcommand{\titredutd}{\textbf{TD12 --- Gaz de fermions}}

\begin{document}


\begin{tcolorbox}[
        colback=gray!20,
        colframe=gray!20,
        width=\dimexpr\textwidth\relax, 
        arc=0pt,outer arc=0pt,
        ]

\texttt{Seules les calculatrices non communicantes et les notes manuscrites personnelles sont autorisées.}

\texttt{Les exercices sont totalement indépendants.}

\texttt{On notera $k_B$ la constante de Boltzmann et $h$ la constante de Planck.}

\end{tcolorbox}



% ______________________________________________________________________________
\section{\medium~Gaz de fermions à deux dimensions}

On étudie un gaz parfait constitué de $N$ fermions indépendants de masse $m$ et de spin $\frac{1}{2}$, libres mais astreints à se déplacer sur une surface d'aire $S$ et à l'équilibre à la température $T$.

\question
Déterminer la densité d'état en énergie $\rho_{2D}(\epsilon)$ (en appliquant des conditions aux limites périodiques pour le vecteur d'onde). 

\question
Rappeler quel est le nombre moyen d'occupation $n(\epsilon)$ d'un état quantique d'énergie $\epsilon$ lorsque le potentiel chimique du gaz est égal à $\mu$.

\question
En déduire la relation implicite qui détermine le potentiel chimique $\mu$ en fonction de $N$, $S$ et $T$.

\medskip

On se place dans l'approximation de la température nulle.

\question
Calculer le niveau de Fermi $\epsilon_F=\mu(T=0)$ et la température de Fermi $T_F=\frac{\epsilon_F}{k_B}$.

\question
Application numérique: le graphite peut être considéré comme étant formé d'un empilement de plans parallèles. Chaque plan est constitué d'un réseau hexagonal à chaque noeud duquel se trouve un carbone ayant un électron de conduction. Le côté de l'hexagone mesure 0,142 nm. Calculer la densité $\frac{N}{S}$ et la position du niveau de Fermi du graphite à température nulle en eV.

\medskip

On se place maintenant à température non nulle.

\question
Calculer l'expression exacte du potentiel chimique à une température quelconque. On remarquera que
$$
\int_0^{+\infty} \frac{dx}{1+A e^x}=\int_0^{+\infty} \frac{e^{-x}dx}{e^{-x}+A}
$$

\question
Tracer la courbe $\mu(T)$ et préciser le comportement de $\mu$ dans la limite des basses températures.

\question
Application numérique: calculer le potentiel chimique pour le graphite lorsque $T=1000$ K. Conclusion ?

\question
Démontrer les relations

$$
J=-E,   C=\frac{2E}{T}- Nk_B\frac{T_F/T}{1-\exp(-T_F/T)}
$$
où $J$ et $E$ sont respectivement le grand potentiel et l'énergie du gaz, et $C$ sa capacité calorifigue (à surface constante et, hélas à nombre de particules constant). La deuxième relation ne peut être obtenue qu'après des calculs assez laborieux.. Qu'on se le dise !


\end{document}
