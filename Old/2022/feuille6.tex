%\documentclass[reponses, utf8, 11pt]{feuille}
\documentclass[utf8, 11pt]{feuille}

\newcommand{\titredutd}{\textbf{TD6 --- Deux approches pour l'oscillateur harmonique}}

\begin{document}


\begin{tcolorbox}[
        colback=gray!20,
        colframe=gray!20,
        width=\dimexpr\textwidth\relax, 
        arc=0pt,outer arc=0pt,
        ]

\texttt{Seules les calculatrices non communicantes et les notes manuscrites personnelles sont autorisées.}

\texttt{Les exercices sont totalement indépendants.}

\texttt{On notera $k_B$ la constante de Boltzmann et $h$ la constante de Planck.}

\end{tcolorbox}



% ______________________________________________________________________________
\section{\medium~Le modèle d'Einstein en micro-canonique}

On considère un solide formé de $N$ ions ou atomes vibrant autour de leurs positions d'équilibre avec la même fréquence $\nu$. On suppose que ce solide est isolé thermiquement et que son énergie est $E$ avec une incertitude que l'on négligera. On rappelle que l'énergie de vibration d'un oscillateur harmonique de fréquence $\nu$ suivant un axe est:
$$
\epsilon_k=(k+\frac{1}{2}) h \nu
$$
où $k$ est un entier naturel. Soit $E \gg \frac{3N}{2} h \nu$, l'énergie associée aux vibrations du solide.

\question
Calculer le nombre de micro-états $\Omega(E,N)$ du système.

\question
En déduire l'expression de l'entropie $S(E,N)$.

\question
Déterminer la température $T$ du solide à l'équilibre. 

\question
Inverser la relation précédente et montrer que l'énergie $E$ s'exprime en fonction de $\beta=\frac{1}{k_BT}$ comme
$$
E=\frac{3}{2} N h \nu + \frac{3Nh \nu}{e^{\beta h \nu}-1}. \nonumber
$$

\question
Comment varie l'énergie à haute température ? En déduire la capacité calorifique du solide dans cette limite. Quelle loi bien connue retrouve-t-on ?



% ______________________________________________________________________________
\section{\medium~Système d'oscillateurs harmoniques classiques}

On considère un système constitué de $N \gg 1$ oscillateurs harmoniques {\it classiques}, de masse $m$, de pulsation propre $\omega$, localisés sur un axe à une dimension et {\it indépendants}.

On rappelle que l'énergie mécanique d'un oscillateur harmonique est :
$$
\epsilon=\frac{m}{2}\omega^2 x^2+\frac{p_x^2}{2m} , \nonumber
$$
où $x$ est l'abscisse de l'écart à la position d'équilibre et $p_x$ la quantité de mouvement associée.

\question
Calculer le volume dans l'espace des phases des états d'énergie inférieure ou égale à $E$. Par des changements de variables appropriés, on fera apparaître le volume $V_{2N}$ de la boule de dimension $2N$ de rayon $R=1$, $V_{2N}=\frac{\pi^N}{\Gamma(N+1)}$.

\question
En déduire le nombre de micro-états $\Phi(E,N)$ d'énergie inférieure ou égale à $E$, puis l'entropie $S(E,N)$ du système. L'entropie est-elle bien extensive ?

\question
Calculer la température de ce système, puis sa capacité calorifique. Comparer aux résultats de l'exercice précédent.

\question
Comment généraliser ce résultat pour une collection d'OH 3D ?





\end{document}
