\elements{
\\
1 : Les atomes étant localisés aux n\oe uds d'un réseau cristallin, ils sont discernables.
\\
2 : $B$.
\\
3 : Les micro-états accessibles ont trois spins $+1$ et deux spins $-1$ (par exemple $\uparrow \downarrow \uparrow \uparrow \downarrow$), $\varOmega$ est le nombre de façons de choisir trois spins parmi cinq (une fois choisis, on leur affecte la valeur $+1$, les deux spins restants valent nécessairement $-1$), soit $ \varOmega = {5 \choose 3} = \frac{5!}{3!\ 2!}=10$ (pour s'en convaincre, on peut facilement recenser les dix micro-états accessibles), $ \langle s_j \rangle = (+1)\frac{6}{10} + (-1) \frac{4}{10} =\frac{1}{5}$, car les 10 micro-états ont la même probabilité $p_m=1/10$ et un spin $j$ vaut $s_j=+1$ dans six micro-états et $s_j=-1$ dans quatre.
\\
4 : $N_-= N - N_+$, donc $ E = - \sum_{j=1}^{N} \mu B s_j = -\Big(+ \mu BN_+ -\mu B(N-N_+) \Big)=\mu B(N-2N_+)$.
\\
5 : $\varOmega (N,E)$ est égal au nombre de façons de choisir $N_+$ spins $+1$ parmi $N$, soit $ \varOmega (N,E)= {N \choose N_+}= \frac{N!}{N_+! (N-N_+)!}$, où $N_+=\frac{N}{2}\Big(1-\frac{E}{N\mu B}\Big)$.
}
