\elements{
\\
1 : $ \int_{-\infty}^{+\infty} {\rm d}v_x \int_{-\infty}^{+\infty} {\rm d}v_y \int_{-\infty}^{+\infty} {\rm d}v_z P(\vec v)= C \Big[\int_{-\infty}^{+\infty} {\rm d}v_x {\rm e}^{-\beta \frac{mv_x^2}{2}} \Big]^3=C \Big[ \sqrt{\frac{2\pi}{\beta m}} \Big]^3= 1$ (normalisation de $P$) voir exercice \ref{exMaths} sur l'intégrale gaussienne, donc $C=\big[\frac{\beta m}{2\pi} \big]^{\frac{3}{2}}$.
\\
2 : Par symétrie :
$F(v_x)= \sqrt{\frac{\beta m}{2\pi}} \,{\rm e}^{-\beta \frac{m v_x^2}{2}}$.
\\
3 : $\langle \vec v \rangle= \vec 0$ car $\langle v_x\rangle =\langle v_y\rangle =\langle v_z\rangle =0$ (isotropie de l'espace, $F$ est une fonction paire).
\\
4 : $v_q=\langle v^2\rangle= \langle v_x^2+v_y^2+v_z^2 \rangle=3 \langle v_x^2\rangle=3 \int_{-\infty}^{+\infty} {\rm d}v_x \, v_x^2F(v_x)=\frac{3}{\beta m}$.
\\
5 : $\langle e \rangle=\frac{1}{2} m \langle v^2 \rangle = \frac{3}{2\beta}=\frac{3}{2}k_B T$.
}
