\elements{
\\
1 : $Z=z^N$ avec pour une molécule $z=\sum_{i=1}^4 {\rm e}^{-\beta \epsilon_i}=1+3{\rm e}^{-\beta \epsilon}$.
\\
 2 : $\langle E\rangle=-\frac{\partial \ln Z}{\partial \beta}=N \epsilon \frac{3{\rm e}^{-\beta \epsilon}}{1+3{\rm e}^{-\beta \epsilon}}$,  $C=\frac{\partial \langle E \rangle}{\partial T}= 3Nk (\beta \epsilon)^2 \frac{{\rm e}^{\beta \epsilon}}{(3+{\rm e}^{\beta \epsilon})^2}$ et $<n_{para}>=\frac{N}{z}=\frac{N}{1+3{\rm e}^{-\beta \epsilon}} $ et $<n_{ortho}>=\frac{3N{\rm e}^{-\beta \epsilon}}{z}=\frac{3N{\rm e}^{-\beta \epsilon}}{1+3{\rm e}^{-\beta \epsilon}}$.
\\
3 : $S= \frac{1}{T}(E-F)$, avec $F=-kT \ln Z$, donc $S=kN \Big[\frac{3 \beta \epsilon }{3+{\rm e}^{\beta \epsilon}} +\ln (1+3{\rm e}^{-\beta \epsilon})\Big] \to kN \ln 4$ quand $\beta \epsilon \ll 1$, car il y a $4^N$ micro-états équiprobables à haute température (et $S \to 0$ quand $\beta \epsilon \gg 1$).
}
