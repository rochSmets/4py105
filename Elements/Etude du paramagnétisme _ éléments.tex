\elements{\\
1 : $z_{cin} = \frac{8 {\cal I} \pi^2}{\beta h^2}$, $z_{mag} = \left( \frac{\sinh(\beta \mu B)}{\beta \mu B} \right) $. Volume accessible ${\cal V} = \frac{4 {\cal I} \pi}{m}$, et $\Lambda_T$ longueur thermique de de Broglie.
\\
2 : $\langle E \rangle = N \langle e \rangle$ et $\langle e_{cin} \rangle = k_B T$, $\langle e_{mag} \rangle = k_B T - \mu B \coth(\beta \mu B)$.
\\
3 : $P(\theta, \phi) = \frac{1}{4 \pi} \frac{\beta \mu B}{\sinh(\beta \mu B)} \sin \theta e^{\beta \mu B \cos \theta}$.
\\
4 : $\langle \mu_x \rangle = \langle \mu_y \rangle = 0$ et $\langle \mu_z \rangle = \mu {\cal L} (\beta \mu B)$.
\\
5 : $z = \frac{\sinh(x(J+\frac{1}{2}))}{\sinh(\frac{x}{2})}$.
\\
6 : $\langle u_z \rangle = g \mu_B J {\cal B}_J(\beta g \mu_B B J)$.
\\
7 : ${\cal B}_J(y) = \tanh(y)$, et $\langle \mu_z \rangle = \mu \tanh( \beta \mu B)$ : résultat classique obtenu pour un système à 2 niveaux.
\\
8 : La loi de Curie énonce que la susceptibilité d'un matériaux pramagnétique est en $\frac{1}{T}$. Or $\chi_m = \mu_0 \partial_B M$ avec l'aimantation $M = n_v \langle \mu \rangle$ où $n_v$ est la densité volumique de moments magnétiques. Le calcul de $\chi_m$ donne la constante de Curie $\farc{\mu_0}{3} n_v (g \mu_B)^2 \frac{J(J+1)}{B}$. La pente à l'origine sur la Fig. est $\chi_m$, d'où la valeur de $J$.
}
