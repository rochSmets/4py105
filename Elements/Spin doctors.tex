\elements{
\\
1 $\epsilon=- \Vec{\mu}.\Vec{B}$,  $=- \mu B$ si parallèles et $=+  \mu B$ si anti-parallèles (1 pt).
\\
2 $E_0=1 \times (-\epsilon)$, $E_1=2 \times (-\epsilon)+ 1 \times \epsilon=-\epsilon$, $E_{1\cup 2}=E_0+E_1=-2\epsilon$ (1 pt).
\\
3 Choisir un spin - parmi les 3 spins de $A_1$ donc 3 micro-états (1 pt).
\\
4 Il faut basculer un spin de $A_1$ de - à + pour assurer la conservation de l'énergie (1 pt) .

Les trois spins de $A_1$ sont donc +: un seul micro-état (1 pt).

5 Il suffit de faire le ratio du nombre de micro-états accessibles dans chacun des cas en raison de l'équiprobabilité de ces micro-états (1 pt). 

Donc $P_-/P_+=1/3$ (1 pt).

6 $E_0=1 \times (-\epsilon)$, $E_1=n \times (-\epsilon)+ (N-n) \times \epsilon=(N-2n)\epsilon$ (1 pt)

$E_{1\cup 2}=E_0+E_1=(N-2n-1)\epsilon$ (1 pt).
\\
7 Choisir $n$ spins + parmi les $N$ spins de $A_1$ donc $C_N^n$ micro-états (1 pt).
\\
8. Il faut basculer un spin de $A_1$ de - à +. Il y a donc $n+1$ spins de $A_1$ dans l'état + (1 pt) 

et donc $C_N^{n+1}$ micro-états (1 pt).
\\
9 $P_-/P_+=C_N^{n+1}/C_N^n = (N-n)/(n+1)$ (1 pt). 

Ce rapport diminue quand $n$ augmente (1 pt).
\\
10 $E_{1\cup 2} = E_0+E_1 = E_0+(N-2n)\epsilon$ (1 pt).
\\
11 Choisir $n$ spins + parmi les $N$ spins de $A_1$ donc $C_N^n$ micro-états (1 pt).
\\
12 Soit $n_r$ le nouveau nombre de spins + de $A_1$. On a $E_{1\cup 2} = E_0+(N-2n)\epsilon = E_r+(N-2n_r)\epsilon$ (1 pt) 

d'où $\Delta n=n_r-n = (E_r-E_0)/(2 \epsilon)$ (1 pt),

qui donne donc $g_r C_N^{n+\Delta n}$ micro-états (1 pt).
\\
13 $P_r/P_0= (g_r C_N^{n+\Delta n})/(g_0 C_N^{n})=g_r/g_0 \times n!/(n+\Delta n)! \times (N-n)!/(N-n-\Delta n)!$ (1 pt).

On peut écrire $n!/(n+\Delta n)!\simeq \exp{[n \ln n - n]}/\exp{[(n+\Delta n) \ln (n+\Delta n) - (n+\Delta n)]} \simeq \exp{[- \Delta n \ln n]}$ (2 pts). 

D'où $P_r/P_0=g_r/g_0 \times \exp{[- \Delta n (\ln n -\ln (N- n) )]}$ (1 pt).
\\
14 En substituant l'expression de $\Delta n$ en fonction de $E_r$,  \mbox{$P_r \propto g_r \exp[- E_r (\ln n - \ln (N- n) )/(2 \epsilon)]$} ( 1 pt), 

donc $\beta=\ln (n/(N-n))/(2 \epsilon)$ (1 pt) 

qui est positive uniquement si $n > N/2$ (1 pt). }
