Dans le modèle de gaz sur réseau, on suppose que les atomes peuvent occuper les $M$ sites d'un réseau cubique simple (de coordinence $q=6$), chaque site pouvant accueillir au plus un atome. L'interaction attractive à courte portée entre les atomes est limitée aux $q=6$ plus proches voisins et est prise en compte via l'hamiltonien :
$$
H_{\mathrm{GR}}=-\epsilon \sum_{\langle i,j\rangle} n_i\, n_j
$$
où le taux d'occupation $n_i$ vaut $1$ si le site $i$ est occupé par un atome, $0$ s'il est vide, et où ${\langle i,j\rangle}$ indique que la somme est prise sur toutes les paires de sites plus proches voisins. La constante $\epsilon$ est positive.  Nous allons étudier ce système dans le cadre du formalisme grand-canonique à la température $T$ et au potentiel chimique $\mu$.