On considère un système constitué de $N \gg 1$ oscillateurs harmoniques \important{classiques}, de masse $m$, de pulsation propre $\omega$, localisés sur un axe à une dimension et \important{indépendants}.

On rappelle que l'énergie mécanique d'un oscillateur harmonique est :
$$
\epsilon=\frac{m}{2}\omega^2 x^2+\frac{p_x^2}{2m} , \nonumber
$$
où $x$ est l'abscisse de l'écart à la position d'équilibre et $p_x$ la quantité de mouvement associée.

\question
Calculer le volume dans l'espace des phases des états d'énergie inférieure ou égale à $E$. Par des changements de variables appropriés, on fera apparaître le volume $V_{2N}$ de la boule de dimension $2N$ de rayon $R=1$, $V_{2N}=\frac{\pi^N}{\Gamma(N+1)}$.

\question
En déduire le nombre de micro-états $\Phi(E,N)$ d'énergie inférieure ou égale à $E$, puis l'entropie $S(E,N)$ du système. L'entropie est-elle bien extensive ?

\question
Calculer la température de ce système, puis sa capacité calorifique. Comparer aux résultats de l'exercice précédent.

\question
Comment généraliser ce résultat pour une collection d'OH 3D ?
