On considère un système de fermions libres et indépendants confinés à la température $T$ et au potentiel chimique $\mu$ dans un volume $V$. Dans l'approximation d'un continuum d'états quantiques individuels, on introduit $\rho(\epsilon)$ la densité d'états en énergie ($0\le \epsilon < +\infty$).

\question Exprimer sous forme d'une intégrale sur l'énergie, le nombre moyen de particules $\langle N \rangle$, l'énergie moyenne $U$ et le grand potentiel $J$.

\question On fait l'hypothèse que $\rho(\epsilon)$ est de la forme $A\epsilon^p$ ($p >0$). Montrer en effectuant une intégration par partie judicieuse que $J=-\frac{U}{p+1}$.

Comme la plupart des métaux usuels, l’argent solide libère un électron de conduction par atome. On considère un volume $V$ d’argent solide contenant un nombre $N$ d’atomes. On donne la masse atomique de l'argent: 108 g.mol$^{-1}$; la masse volumique de l'argent: 10,5 g.cm$^{-3}$; la masse de l'électron : 9,1 $\times 10^{-31}$ kg.

\question Calculer la densité de particules $n=\frac{N}{V}$ de l'argent solide.


\question On rappelle qu'à trois dimensions, la densité d'état d'un gaz parfait de fermions de spin $1/2$ et de masse $m$ est égale à $\rho(\epsilon)=\frac{V}{2\pi^2}(\frac{2 m}{\hbar^2})^{\frac{3}{2}} \sqrt{\epsilon}$. \'Etablir l’expression de la température de Fermi $T_F$ du gaz formé par les électrons de conduction supposés sans interaction en fonction de $n$ et des données. 

\question Calculer l’ordre de grandeur de $T_F$ pour l’argent. Quelle est l’importance physique de ce résultat ?  Peut-on traiter les électrons d'un métal avec la statistique de Maxwell-Boltzmann à température ambiante ?

\question \'Etablir  l’expression de la pression $P$ de ce gaz de Fermi à température nulle. Calculer l’ordre de grandeur de $P$ pour l’argent. Pourquoi les électrons ne s’échappent-ils pas du solide sous l’effet de cette pression ?
