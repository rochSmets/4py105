Un gaz parfait est constitué de $N$ molécules statistiquement indépendantes et uniformément réparties en moyenne dans un récipient de volume $V$.  Soit $k$ le nombre (aléatoire) de molécules contenues dans un sous-volume $v$ du récipient.

\medskip

\question
Quelle est la valeur moyenne $\langle k \rangle$ de $k$ ?

\question
Quel est l'écart-type $\sigma_k$ de $k$ ? \\
Indice : on peut écrire la variable $k$ comme une somme de $N$ variables aléatoires indépendantes.
  
\donnees{$v=\frac{V}{2}$ et $N=100$, puis $N=10^{10}$ et $N={\cal N}_A$.}

\question
Faire l'application numérique

\question
Pour $N$ très grand et $\frac{v}{V}$ fixé, vers quelle loi tend la distribution de probabilité $P(k)$ de $k$ ?

\question
Quelle est la probabilité que toutes les molécules du gaz soient dans le volume $v$ ?

On veut calculer la probabilité exacte $P(k)$ qu'il y ait $k$ molécules dans le volume $v$.

\question
De combien de manières différentes peut-on choisir les $k$ molécules parmi $N$ qui sont dans le volume $v$ ?

\question
Quelle est la probabilité de \emph{l'un} de ces choix ? (par exemple, pour $k=4$ et $N=100$, quelle est la probabilité que les particules numéros 8, 12, 35 et 42, par exemple, soient dans le volume $v$ ?)

\question
En déduire l'expression de $P(k)$. Quel est le nom de cette distribution de probabilité ?

\question
On rappelle la formule du binôme de Newton
\begin{equation} 
(x+y)^N=\sum_{n=0}^N \binom Nn x^n y^{N-n}.
\label{sumbinom}
\end{equation}
Vérifier que la distribution de probabilité $P(k)$ est bien normalisée.

\question
Calculer les dérivées première et seconde de l'égalité (\ref{sumbinom}) par rapport à $x$, \emph{puis} remplacer $y$ par $1-x$ dans les expressions obtenues. Utiliser les formules ainsi obtenues pour retrouver la moyenne et la variance de $k$.

\question
On se place à la limite thermodynamique ($N \to \infty$, $V \to \infty$ tels que la densité $\frac{N}{V}$ est constante). En considérant le nombre de particules comme une variable continue, montrer en utilisant la formule de Stirling que la distribution de probabilité de $k$ se comporte comme une loi gaussienne au voisinage de $\langle k \rangle$ (on posera $k = \langle k \rangle + s$ avec $s \ll N$). Ce résultat est-il surprenant ?
