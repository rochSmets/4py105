\question
Rappeler les expressions de l'entropie molaire $S(E,V,{\cal N}_A)$ d'une mole de gaz parfait monoatomique (atomes de masse $m$) confinée dans un volume $V$ et d'énergie $E$ et du nombre de micro-états $\phi(E,V,{\cal N}_A)$ d'un tel gaz d'énergie $\leq E$. 

\question En déduire la relation entre la température $T$ (micro-canonique) et l'énergie $E$ du gaz. Ré-exprimer $S$ en fonction des variables $\rho=\frac{{\cal N}_A}{V}$ et $T$. On introduira la longueur d'onde thermique $\Lambda_T$ de de Broglie définie par
$$\Lambda_T=\frac{h}{\sqrt{2 \pi m k_B T}}$$.

\question Calculer l'entropie d'une mole d'argon aux conditions normales de température (25$^\circ$C) et de pression (100 kPa).
On considérera que sous ces conditions, l'argon peut être traité comme un gaz parfait.

\question Quelle est en Joule l'énergie d'une mole de gaz parfait monoatomique à  25$^\circ$C ? Calculer l'augmentation (en pourcentage) de $\phi(E,V,{\cal N}_A)$ et la variation correspondante de $S(E,V,{\cal N}_A)$  lorsque on accroît l'énergie respectivement de 58 neV, 39 meV, 13.6 eV, 938 MeV (un bonus pour celles et ceux qui donnent un sens physique à ces valeurs). 

\question Conclure. On pourra estimer de combien varie l'énergie gravitationnelle entre le gaz et vous lorsque, situé à la distance de 10 mètres, vous reculez de 10 cm.

\question Tracer sommairement $S$ en fonction de $T$ à $\rho$ fixée. Pourquoi le résultat est-il nécessairement incorrect à basse température ? Quelle est l'origine du problème ? Donner la valeur de $\rho \Lambda_T^3 $ caractéristique de ce défaut et la valeur correspondante de la température pour l'argon sachant qu'on a refroidi une mole de ce gaz à partir des conditions normales de température et de pression.
