Soit un système de $N$ spins $\frac12$ portés par des atomes localisés
aux n\oe uds d'un réseau cristallin et plongés dans un champ
magnétique $\mathbf{B}$. On néglige les interactions entre les
spins. Ce système de particules {\it indépendantes} est isolé à l'énergie
$E$, donnée par
$$
E = - \sum_{j=1}^{N} \mu B s_j \quad \textrm{avec} \quad s_j = \pm
1 \enspace .
$$

\question
Les atomes sont-ils discernables ou indiscernables ? 

\question
Quels sont les paramètres extérieurs ? 

\question
Pour commencer considérons $N=5$ spins isolés à l'énergie
$E=-\mu B$. Quel est le nombre $\varOmega$ de micro-états accessibles
au système ? Que vaut la moyenne $\langle s_j \rangle$ d'un spin ? 

\medskip

Généralisons au cas d'un système de $N$ spins indépendants à l'énergie $E$.

\question
Soit $N_+$ le nombre de spins parallèles au champ magnétique ($s_j=+1$) et
$N_-$ le nombre de spins antiparallèles au champ magnétique ($s_j=-1$).
Exprimer  $E$ en fonction de $N$ et $N_+$.  

\question
Quel est le nombre de micro-états accessibles $\varOmega (N,E)$ ?
