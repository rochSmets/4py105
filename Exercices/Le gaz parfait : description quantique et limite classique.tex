\tiret{Une particule quantique dans une boîte}

Une particule de masse $m$ (sans spin) est confinée dans une enceinte cubique de dimension linéaire $L$ et de volume $V=L^3$. Son énergie est donnée par
\begin{equation} \label{eqNiveauEnergie1parti}
E=\frac{\hbar^2 \pi^2}{2mL^2} (n_{x}^{2}+n_{y}^{2}+n_{z}^{2})
\end{equation}
où $n_x$, $n_y$ et $n_z$ sont trois nombres entiers positifs associés aux trois degrés de liberté de la particule.

\medskip
On cherche à évaluer sommairement la façon dont le nombre de micro-états $\varOmega(V,E)$ varie avec $E$ et $V$. Pour cela, on se place dans l'approximation des grands nombres quantiques, de telle façon que l'énergie varie quasi continûment avec les nombres quantiques associés. La fonction $\varOmega(V,E)$ est alors elle-même une fonction presque continue de $E$.

\question
On considère tout d'abord le cas d'une particule dans une boîte à \important{une} dimension de taille $L$.  \`A partir de l'expression (\ref{eqNiveauEnergie1parti}) des niveaux d'énergie de la particule évaluer le nombre d'états $\Phi(L,E)$ d'énergie inférieure ou égale à $E$.  En déduire l'expression de la densité $\rho(L,E)$ d'états compris entre les énergies $E$ et $E + \delta E$ avec $\delta E\ll E$, ainsi que $\varOmega(L,E)$. 

\question
Obtenir la densité d'états $\rho(L,E)$ en {\it deux} puis {\it trois} dimensions quantiquement.
	
\question Calculer le nombre de micro-états accessibles pour un atome d'argon de masse molaire $M=40$ g.mol$^{-1}$ d'énergie comprise entre $E$ et $E+\delta E$, où $E=6 \times 10^{-21}$ J et $\delta E = 10^{-31}$ J, dans un volume d'un litre.

\medskip

\tiret{Le gaz parfait quantique}
	
L'enceinte contient $N$ particules sans interaction et supposées {\it discernables}. Malgré cette hypothèse, nous allons étudier ce système dans le cadre de la mécanique quantique.
	
\question
Montrer que ce gaz parfait est équivalent à une particule évoluant dans un espace à $3N$ dimensions. Calculer $\Phi(N,V,E)$ et $\rho(N,V,E)$ en vous inspirant du cas d'une particule seule.

\question
En déduire l'entropie de ce gaz.

\question
Calculer la température et la pression du gaz parfait. Vérifier que vous retoruver l'équation d'état bien connue.
