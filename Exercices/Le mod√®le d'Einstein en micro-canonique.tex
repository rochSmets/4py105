On considère un solide formé de $N$ ions ou atomes vibrant autour de leurs positions d'équilibre avec la même fréquence $\nu$. On suppose que ce solide est isolé thermiquement et que son énergie est $E$ avec une incertitude que l'on négligera. On rappelle que l'énergie de vibration d'un oscillateur harmonique de fréquence $\nu$ suivant un axe est:
$$
\epsilon_k=(k+\frac{1}{2}) h \nu
$$
où $k$ est un entier naturel. Soit $E \gg \frac{3N}{2} h \nu$, l'énergie associée aux vibrations du solide.

\question
Calculer le nombre de micro-états $\Omega(E,N)$ du système.

\question
En déduire l'expression de l'entropie $S(E,N)$.

\question
Déterminer la température $T$ du solide à l'équilibre. 

\question
Inverser la relation précédente et montrer que l'énergie $E$ s'exprime en fonction de $\beta=\frac{1}{k_BT}$ comme
$$
E=\frac{3}{2} N h \nu + \frac{3Nh \nu}{e^{\beta h \nu}-1}. \nonumber
$$

\question
Comment varie l'énergie à haute température ? En déduire la capacité calorifique du solide dans cette limite. Quelle loi bien connue retrouve-t-on ?
