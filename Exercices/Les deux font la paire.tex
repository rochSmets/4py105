On considère un gaz sur réseau constitué de $N$ particules pouvant se fixer sur $N$ sites d'un réseau. Chaque site peut comporter au plus deux particules. On appelle $N_0$, $N_1$ et $N_2$ le nombre de sites contenant respectivement zéro, une ou deux particules. Chaque fois qu'il y a deux particules sur un même site, le système gagne une énergie d’interaction $\epsilon$. 

\question Quelles relations y-a-t-il entre $N_0$, $N_1$ et $N_2$ ? Exprimer $N_0$ et $N_1$ en fonction de $N_2$.

\question Pour $\epsilon > 0 $, quel est l’état fondamental du système et quelle est son énergie ? Quel est l’état d’énergie maximale du système et que vaut cette énergie ? Mêmes questions pour $\epsilon < 0 $. Lequel des deux cas $\epsilon > 0 $ ou $\epsilon < 0 $ correspond à une situation où les particules se repoussent ?

\question On fixe $E$ et $N$. Calculer le nombre de micro-états $\Omega=\Omega(E,N)$ accessibles d'énergie $E$. En déduire l'entropie du système $S(E,N)$ puis la température micro-canonique $T$ du système.

\question Exprimer $N_2$ puis $E$ en fonction de $T$ et de $N$.  

\question On se place dans le cas où $\epsilon > 0 $. Tracer sommairement $E$ en fonction de $T$.

\question Interpréter ce qui se passe à basse température puis à haute température.

\question On suppose maintenant que les particules ont un spin $\frac{1}{2}$. Une particule seule sur un site a donc deux orientations possibles (up ou down). En revanche, quand il y a deux particules sur le même site, le principe d’exclusion de Pauli implique qu'il n’y a qu'un seul choix: une des deux particules a le spin up et l'autre le spin down. Quelle sont les nouvelles expressions de $\Omega$ et de $N_2=N_2(T,N)$ ?
