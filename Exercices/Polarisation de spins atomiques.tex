Soit une substance dont les atomes sont magnétiques : ils possèdent un moment magnétique élémentaire (de spin) $\mu_0 \simeq 10^{-23}$ J.T$^{-1}$. A cause du caractère quantique du phénomène, la mesure de la projection de ce moment sur un axe ne peut prendre que deux valeurs $\pm \mu_0$. Pour polariser ces atomes (c'est-à-dire faire aligner leurs moments magnétiques) quand ils sont dans un gaz peu dense, on applique un fort champ magnétique $B$ et on refroidit à une température absolue $T$ suffisamment basse. Expliquer ces choix à la lumière de vos connaissances sur le facteur de Boltzmann. On pourra s'appuyer sur ce qui suit.
On produit couramment en laboratoire des champ magnétiques d'environ  5 T. \`A quelle température a-t-on trois fois plus de moments magnétiques parallèles au champ magnétique (supposé uniforme) que de moments magnétiques anti-parallèles au champ ? Il vous faudra exprimer le rapport entre la probabilité d'avoir un moment parallèle au champ et la probabilité d'avoir un moment anti-parallèle au champ en fonction de $\mu_0, k_B, B$ et $T$.
