On se propose de déterminer l'expression de la capacité calorifique d'un gaz de molécules diatomiques  hétéro-nucléaires (avec deux atomes différents). On admet que l'Hamiltonien d'une telle molécule diatomique peut s'écrire sous la forme de trois termes principaux $h = h_{tr} + h_{rot} + h_{vib}$. On a respectivement

\begin{itemize}

\item 
$h_{tr} =\frac{\vec{P}^2}{2M}$ où $\vec{P}$ est la quantité de mouvement de la molécule et $M$ sa masse

\item
$h_{rot} =\frac{\vec{l}^2}{2I}$ où $\vec{l}$ est le moment cinétique orbital caractérisant la rotation de la molécule et $I$ le moment d'inertie correspondant (par rapport à  un axe passant par le centre de gravité de la molécule et perpendiculaire à  l'axe rejoignant les deux atomes)

\item
$h_{vib} =\frac{p_x^2}{2\mu}+\frac{k x^2}{2}$, hamiltonien d'un oscillateur harmonique à  une dimension ($x$ est l'écart à  la distance d'équilibre entre les deux molécules et $p_x$ l'impulsion correspondante; $\mu$ est ici la masse réduite des deux atomes).

\end{itemize}


Les valeurs propres de cet Hamiltonien, dans l'approximation du découplage entre ces différentes formes d'énergie, sont telles que l'énergie de la molécule s'écrit $\epsilon =\epsilon_{tr}+\epsilon_{rot}+\epsilon_{vib}$.

On considère un gaz très dilué de $N$ molécules diatomiques identiques à  la température $T$ dans le volume $V$. On suppose que la fonction de partition totale $Z=Z(N,V,T)$ peut s'écrire $Z= z^N/N!$ où $z=z(V,T)$ est la fonction de partition correspondant à  une seule molécule.

\question{Donner trois exemples de gaz diatomiques hétéro-nucléaires.}

\question{Préciser le sens physique de la factorisation $Z= z^N/N!$.}

\question{Rappeler dans l'ensemble canonique la relation entre $Z$ et l'énergie moyenne $\langle E \rangle$ du système. En déduire que la capacité calorifique à  volume constant $C_V$ est donnée par $\frac{C_V}{k_B}=\beta^2 \left( \frac{\partial^2 \ln Z}{\partial \beta^2} \right)_V$.} 

\question{Montrer que, dans les conditions de l'énoncé, $z$ peut s'écrire $z=z_{tr}z_{rot}z_{vib}$ en explicitant le sens de chaque terme. En déduire que l'énergie moyenne et  la capacité calorifique du gaz peuvent s'écrire sous la forme de sommes de trois termes,
associés aux trois différentes formes de l'énergie.}

\question{Combien de degrés de liberté quadratiques possède cette molécule ? En utilisant le théorème d'équipartition que l'on énoncera, que prédit-on dans la limite classique pour l'énergie moyenne et la capacité calorifique par molécule ?}


{\sffamily\bfseries{L'énergie cinétique de translation}}

Aux températures considérées, on peut effectuer un calcul classique pour estimer $z_{tr}$.

\question{Donner sous forme intégrale l'expression de $z_{tr}$. On introduira $h$ la constante de Planck.}

\question{On rappelle que $\int_{-\infty}^{+\infty} \D u \exp(-u^2)=\sqrt{\pi}$. Montrer que $z_{tr}=\frac{V}{\Lambda^3}$ où $\Lambda$ est une grandeur que l'on déterminera en fonction de $M$, $h$ et $k_BT$.}

\question{En déduire l'énergie moyenne $e_{tr}$ par molécule et la capacité calorifique (à  volume constant) correspondante $c_{tr}$. Comment cette dernière dépend-t-elle de la température ?}


{\sffamily\bfseries{L'énergie de vibration}}

\question{Rappeler l'expression des différents niveaux d'énergie discrets d'un oscillateur harmonique quantique à  une dimension. L'énergie du niveau fondamental est $\frac{\hbar \omega}{2}$ et on introduit la température de vibration associée $\Theta_{\mathrm{vib}} = \frac{\hbar \omega}{2 k_B}$.}

\question{Montrer que $z_{vib}=\frac{1}{2 \sinh(\frac{\hbar \omega}{2 k_B T})}$.}

\question{En déduire l'énergie moyenne $e_{vib}$ par molécule et la capacité calorifique (à  volume constant) correspondante $c_{vib}$. Comment cette dernière dépend-t-elle de la température ? On pourra étudier les limites basses températures et hautes températures.}


{\sffamily\bfseries{L'énergie de rotation}}

Les niveaux d'énergie d'une molécule di-atomique en rotation considérée comme un rotateur rigide sont donnés
par l'expression $\epsilon_{rot}=\frac{\hbar^2}{2I}J(J+1)$ où $0 \le J \le + \infty$ est le nombre quantique (entier) de rotation. Il y a $2J+1$ états dégénérés pour cette valeur de $J$. Il est utile d'introduire la température caractéristique $\theta_{\mathrm{rot}}=\hbar^2/(2Ik_B)$ et le rapport $x =\frac{\theta_{\mathrm{rot}}}{T}$ pour alléger les notations.

\question{\'Ecrire l'expression de la fonction de partition $z_{rot}$ (sans la calculer).}

\question{Dans la limite haute température, on peut développer $z_{rot}$ sous la forme $z_{rot}=\frac{1}{x} \left( 1+\frac{x}{3}+\frac{x^2}{15}+\ldots \right)$. En déduire l'énergie moyenne $e_{rot}$ par molécule et la capacité calorifique (à  volume constant) correspondante $c_{rot}$ dans cette limite. Montrer alors que $c_{rot}$ décroît avec la température.}

\question{Dans la limite basse température, seuls les niveaux de basse énergie seront peuplés. Montrer en conservant le nombre de termes adéquats dans la fonction de partition que, dans cette limite, $c_{rot}=12 k_B x^2 \exp(-2x)$.}

\question{En déduire l'allure probable de $\frac{c_{rot}}{k_B}$ en fonction de $\frac{T}{\theta}$.}

\question{Déduire de toutes les questions précédentes l'expression de la capacité calorifique à  volume constant ${\cal C}_V$ dans la limite haute température.}



\subsection{Application au HD}

Le deutérure d'hydrogène est formé d'un proton H et d'un atome de deutérium D. On a dans ce cas $\theta_{\mathrm{rot}}=64$ K et $\frac{\hbar \omega}{k_B}=5382$ K. Calculer ${\cal C}_V$ à  l'ambiante (300 K) et à  3000 K.

%\section{Une condensation à  deux dimensions ?}
%On considère un modèle de fluide  à  deux dimensions, surface $S$, température $T$ et potentiel chimique $\mu$ comme un \og gaz \fg \ de bosons  indépendants de spin nul, de masse $m$ et dont les niveaux d'énergie  sont donnés par
%\begin{align*}
%\epsilon (k)=\frac{\hbar^2 k^2}{2m},
%\end{align*}
%où $k$ est le vecteur d'onde.
%
%\question{Rappeler l'expression du nombre moyen d'occupation $n(\epsilon)$ d'un micro-état d'énergie $\epsilon$.}
%
%\question{Montrer que la densité en énergie des micro-états s'écrit sous la forme $\rho (\epsilon)=A$ où le calcul de $A$ fait appel à  la quantification du vecteur d'onde dans la boîte de surface $S=L_xL_y$. On utilisera des conditions aux limites périodiques pour exprimer les valeurs autorisées de $\vec{k}$.}
%
%\question{En déduire les expressions formelles du nombre moyen $\langle N \rangle$ de particules et de leur énergie moyenne $\langle E \rangle$ sous forme intégrale, en fonction de $T, S$ et $\mu$ et des données. On introduira $\alpha=N\frac{\Lambda^2}{S}$ où $\Lambda=\frac{h}{\sqrt{2\pi m k_B T}}$. }
%
%\question{On donne l'intégrale $\int_0^{+\infty} \frac{dx}{e^{x-y}-1}=-\ln \left(1-e^y \right)$. Calculer explicitement $\mu $ en fonction de $k_BT$ et de $\alpha$.}
%
%\question{En déduire que, dans le cas d'un fluide de bosons bidimensionnel, il n'y a pas de condensation de Bose-Einstein.}

