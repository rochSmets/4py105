On considère un gaz parfait constitué de $N (\sim {\cal N}_A)$ bosons sans spin ultra-relativistes, {\bf indiscernables}, confinées sur une surface $A$ de dimension 2 à la température $T$ et au potentiel chimique $\mu$: l'énergie cinétique $\epsilon_i$ d'une particule $i$ est donnée par $\epsilon_i= p_ic$ où $c$ est la vitesse de la lumière et $p_i=|\Vec{p_i}|=\hbar |\Vec{k_i}|$ la norme de sa quantité de mouvement $\Vec{p_i}$ associée au vecteur d'onde $\Vec{k_i}$. On admet que la densité d'état dans l'espace réciproque (des vecteurs d'onde) est $\hat{\rho}(\Vec{k})d^2 \Vec{k}= \frac{A}{(2 \pi)^2} d^2 \Vec{k}$.

\question Montrer que la densité d'état en énergie $\rho_E(\epsilon)$ est de la forme $K \epsilon$ où $K$ s'exprime en fonction de $A, h$ et $c$. 

\question Rappeler l'expression du nombre moyen de particules $n_i^B(\epsilon)$ dans un état quantique $i$ d'énergie $\epsilon$ en fonction de $\beta, \epsilon$ et $\mu$.

\question \'Ecrire les expressions formelles dans l'approximation continue (sous forme d'intégrales sur l'énergie)  du nombre moyen de particules $N$ et de l'énergie moyenne $U$. On introduira $\alpha=N\frac{\Lambda^2}{A}$ où $\Lambda=\frac{h}{\sqrt{2\pi m k_B T}}$.

\question
Conclure quant à l'existence, ou l'absence d'un phénomène de condensation de Bose-Einstein pour des bosons ultrarelativistes en deux dimensions. On rappelle que
\begin{equation}
g_n(f)=\frac{1}{\Gamma (n)} \int_0^{+\infty} \frac{x^{n-1}\ dx}{\frac{\exp x}{f}-1}=\sum\limits_{p=1}^{+\infty} \frac{f^p}{p^n}.
\end{equation}
