On considère un système de sept atomes. Le moment magnétique $\mu_i$ de chaque atome $i=1,2,\dots, 7$ fluctue et peut valoir $+\mu$ ou $-\mu$. Les interactions entre les moments magnétiques sont négligées, on suppose le champ magnétique extérieur nul. Le système est composé de deux sous-systèmes: le sous-système~I qui contient quatre atomes et le sous-système~II qui en contient trois. On suppose que tous les micro-états du système sont équiprobables

\question
Calculer le nombre $\varOmega$ de micro-états accessibles du système.
  
\question
Calculer la probabilité $Pr(M_I=-2\mu)$ pour que $M_I=-2\mu$, où $\displaystyle M_I=\sum_{i=1}^4 \mu_i$ est le moment total du sous-système~I. 

\question
Calculer la valeur moyenne $\langle M_I \rangle$ de $M_I$. 

\medskip

On suppose maintenant que l'on impose un moment magnétique total
$M_I+M_{II}$ égal à $\mu$. Tous les micro-états qui vérifient cette
contrainte sont équiprobables.

\question
Calculer le nouveau nombre $\varOmega$ de micro-états accessibles du système. 

\question
Calculer la nouvelle probabilité $Pr(M_I=-2\mu)$  pour que $M_I$=-2$\mu$. Quelle est la probabilité $Pr(M_{II}=3\mu)$ que $M_{II}=3\mu$ ? 

\question
Calculer la nouvelle valeur moyenne $\langle M_I \rangle$ de $M_I$. 
