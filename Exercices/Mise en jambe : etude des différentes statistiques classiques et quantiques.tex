On considère un système constitué de deux particules disposées sur deux niveaux d'énergies respectives $\epsilon_1=-\epsilon$ et $\epsilon_2=\epsilon$, de dégénérescences 1, à  l'équilibre avec un thermostat à  la température $T$.
\'Enumérer toutes les configurations microscopiques envisageables et en déduire successivement la fonction de partition et l'énergie moyenne $\bar{\epsilon}(T)$ du système dans les cas suivants:
\begin{itemize}
\item les deux particules sont discernables;
\item les deux particules obéissent à  la statistique de Bose-Einstein;
\item les deux particules obéissent à  la statistique de Fermi-Dirac;
\item les deux particules sont des fermions de spin 1/2.
\end{itemize}
On pourra introduire la grandeur $\zeta=\exp(-\beta \epsilon)$ ou faire appel à  des fonctions trigonométriques (hyperboliques) bien connues pour expliciter les résultats. Sur un graphique représentant $\bar{\epsilon} (T)$, vous placerez, pour les 4 systèmes étudiés, les valeurs limites de l'énergie moyenne pour $T \rightarrow 0$ et $T \rightarrow +\infty$. Vous proposerez ensuite la forme qualitative des chacune de ces 4 courbes en vous interrogeant notamment sur leurs comportements en $T \rightarrow 0$ et $T \rightarrow +\infty$.
