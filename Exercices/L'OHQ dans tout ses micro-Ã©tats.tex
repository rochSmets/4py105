On considère dans un premier temps une collection de $N $ oscillateurs harmoniques quantiques 1{\sc d} identiques, faiblement couplés entre eux et isolés du reste de l'univers. Le nombre de quanta d'énergie qu'ils se partagent est égal à $M$. Dans toute la suite, on considérera  que $N \gg 1$ et $M \gg 1$.

\question
Calculer l'énergie totale $E(N,M)$ et le nombre de micro-états $\Omega(N,M)$ du système, puis $\ln \Omega(N,M)$.

\question
Soit un {\sc ohq} particulier. Calculer le nombre de micro-états $\Omega(N,M|m)$ du système pour lequel cet {\sc ohq} possède exactement $m$ quanta d'énergie. En déduire la probabilité $p(m)$ qu'un {\sc ohq} contienne exactement $m$ quanta d'énergie.

\question
On introduit $\overline m= \frac{M}{N}$. Quelle est l'interprétation de $\overline m$ ? Montrer que dans l'hypothèse où $M \gg m$, $p(m)$ peut être approché par
$$
p(m) \sim \frac{1}{1+\overline m} \, \left(\frac{\overline m}{1+\overline m}\right)^m
$$
Tracer $p(m)$. Montrer que même dans l'approximation précédente $\sum_{m=0}^{m=M}p(m)=1$.

\medskip
On considère désormais deux collections d'{\sc ohq}, appelées 1 et 2,  comportant $N_1$ et $N_2$ {\sc ohq} respectivement et $M_{1i}$ (resp. $M_{2i}$) quanta d'énergie. Initialement isolées, elles sont mises en contact thermique à l'instant initial pour former une collection unique de $N=N_1+N_2$ {\sc ohq} faiblement couplés avec $M=M_{1i}+M_{2i}$ quantas. Tous ces nombres sont $\gg 1$.

\question
Calculer le nombre de micro-états du système réuni juste avant le contact $\Omega_i$ puis juste après $\Omega_f$. Comparer $\Omega_i$  et $\Omega_f$.

\question
Grâce au contact et au faible couplage, le nombre de quanta de 1 est désormais libre de fluctuer par échange d'énergie avec 2. Exprimer littéralement la probabilité $P(M_1)$ pour que la collection 1 possède $M_1$ quantas exactement.  

\question
En vous aidant de la formule de Stirling pour écrire $\ln P(M_1)$, calculer la valeur $\overline {M_1}$ de $M_1$ qui rend $P(M_1)$ maximum. Interpréter le résultat obtenu.

\question
Calculer le nombre $\Omega_e$ de micro-états du système correspondant à  $\overline{M_1}$ quanta dans 1. Comparer $\Omega_e$  à $\Omega_i$  et $\Omega_f$ , puis $\ln \Omega_e$  à $\ln \Omega_i$  et $\ln \Omega_f$. Que constate-t-on avec bonheur ?

\question
Expliquer (avec des mots) ce qu'il advient quand on met en contact thermique deux systèmes.
