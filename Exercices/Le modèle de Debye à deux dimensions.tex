Dans un solide, chaque onde associée aux vibrations des atomes du réseau cristallin, de vecteur d'onde $\Vec{k}$ et de pulsation $\omega$, a l'énergie $\epsilon= \hbar \omega$. Ces ondes peuvent se représenter par une assemblée à nombre indéterminé (c'est-à-dire à potentiel chimique nul) de quasi-particules de nature bosonique, libres et indépendantes, appelées phonons pour lesquelles l'impulsion $\Vec{p}$ est donnée par $\Vec{p}=\hbar \Vec{k}$. Pour les petits vecteurs d'onde, ces ondes présentent des modes acoustiques dans lequel on a linéarité entre le module du vecteur d'onde et la pulsation. 

Nous considérons dans la suite le cas d'un réseau plan monoatomique de $N$ atomes à deux dimensions de longueurs $L_x$ et $L_y$. On pose $S=L_x L_y$. Il existe un mode acoustique longitudinal de célérité $c_l$ telle que $\omega = c_l k $ et un mode acoustique transversal de célérité $c_t$ telle que $\omega = c_t k $.

Le modèle de Debye consiste à attribuer à chacune des deux polarisations possibles des ondes de vibrations, des célérités respectives $c_l$ et $c_t$, constantes dans tout le domaine de variation de $k$ (et non plus aux \mbox{\og petits \fg \  $k$}). 

\question  Déterminer la densité d'état en énergie $\rho_{2D}(\epsilon)$ de ces phonons, en appliquant des conditions aux limites périodiques pour le vecteur d'onde. Montrer qu'elle est de la forme $\rho_{2D}(\epsilon)=A \epsilon$. On introduira $c$ telle que

$$\frac{2}{c^2}=\frac{1}{c_l^2}+\frac{1}{c_t^2}$$.

(En admettant le résultat $\rho_{2D}(\epsilon)=A \epsilon$, on peut poursuivre l'exercice.)

\question Le nombre total de mode de vibrations $\int \rho_{2D}(\epsilon) d\epsilon$ étant fixé à $2 N$ (2 polarisations  $\times$ $N$ particules), montrer que l'énergie d'un phonon doit avoir dans ce modèle une borne supérieure $\epsilon_D$. On introduit la température caractéristique $\Theta=\frac{\epsilon_D}{k_B}$. Relier $A$ à $N$, $\Theta$ et $k_B$.

\question \'Ecrire sous forme intégrale l'énergie interne $U$ en fonction de $N$, $\Theta, T$ et $k_B$.  Montrer que

$$U=4 N k_B \Theta \left( \frac{T}{\Theta} \right)^3  \int_0^{\frac{\Theta}{T}} \frac{x^2 dx}{e^x-1}.$$

\question \'Etudier les limites basse température ($T \ll \Theta$) et haute température ($T \gg \Theta$) de $U$. On introduira le nombre $I=\int_0^{+\infty} \frac{x^2 dx}{e^x-1}$.
 

\question En déduire les limites basse température ($T \ll \Theta$) et haute température ($T \gg \Theta$) de la capacité calorifique $C_S$. Commenter la limite obtenue à haute température. Tracer la courbe représentative de $\frac{C_S}{N k_B}$ en fonction de $\frac{T}{\Theta}$.

\question Existent-t-ils des situations physiques qui correspondent à notre modèle ?
