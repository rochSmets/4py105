On considère un système composé d'un grand nombre $N$ de particules, dont les interactions sont négligeables entre elles. Chacune des particules possède quatre niveaux d'énergie \og interne \fg \, respectivement d'énergies $-3\epsilon$, $-\epsilon$,  $+ \epsilon$ et  $+3 \epsilon$.  Les niveaux d'énergie $-\epsilon$,  et $+\epsilon$ ont une dégénérescence $g=3$.

\

\tiret{Etude dans l'ensemble canonique}

 On suppose que le système est à l'équilibre avec un thermostat à la température $T$.

 \question Sans calcul mais en justifiant, donner l'énergie moyenne par particule à \og basse température \fg puis à \og haute température \fg. On précisera le sens de basse et haute température ici.
 
\question Calculer $z(\beta)$ la fonction de partition d'une particule et montrer qu'elle se factorise en utilisant une identité remarquable avec $(a+b)^3$.

\question
En déduire que l'énergie  moyenne $u$ par particule est égale à $u=- 3\epsilon \tanh (\beta \epsilon)$.

\question
En déduire la contribution $c$ de chaque particule à la capacité calorifique de ce système. Tracer grossièrement $c$ en fonction de la température.

\


\tiret{\'Etude dans l'ensemble micro-canonique}

On suppose que le système est isolé et a une énergie donnée $E$.

\question Sur quel domaine peut varier $E$ ? Donner sa valeur minimale et sa valeur maximale.

\question Donner les deux relations qui existent entre $N$, $E$ et les populations des niveaux $N_{+3}$, $N_{-3}$, $N_{+1}$ et $N_{-1}$ (respectivement d'énergie $3 \epsilon$, $-3 \epsilon$, $\epsilon$ et $-\epsilon$ ).

\question Exprimer le nombre de micro-états $\Omega(N_{+3},N_{-3},N_{+1},N_{-1})$ qui correspondent à un jeu donné de valeurs de $N_{+3}$, $N_{-3}$, $N_{+1}$ et $N_{-1}$.

\question Que faudrait-il faire \og en théorie \fg pour obtenir $\Omega(E,N)$, le nombre de micro-états du système à $N$ particules et d'énergie $E$, à partir des réponses aux deux questions précédentes ?

\question Donner succinctement mais précisément les grandes étapes du calcul dans la limite thermodynamique de l'entropie microcanonique $S=k_B \ln \Omega(E,N)$. 