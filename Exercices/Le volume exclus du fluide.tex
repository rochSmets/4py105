On considère dans cet exercice le cas $\epsilon=0$.

\medskip

\question
En quoi ce modèle inclut-il bien l'effet de coeur dur de l'interaction atomique ?
 
\question
Que vaut la fonction de partition $Z(N,T)$ pour $N$ particules d'un tel fluide  ($N<M$)?

\question
Montrer que la grande fonction de partition $\Xi(\mu,T)$ du fluide se factorise comme le produit des grandes fonctions de partition $g$ d'un site, que l'on exprimera.

\question
Calculer la pression $P(\mu,T)$ du système ainsi que le nombre moyen $N(\mu,T)$ d'atomes. 

\question
Déterminer l'équation d'état du fluide en fonction du taux moyen d'occupation d'un site, $n=\frac{N}{M}$, puis en fonction de la densité $\rho=\frac{N}{V}$ (en introduisant $v_0=V/M$ que l'on interprétera).  Que devient cette équation dans la limite des faibles densités ?
