On considère un gaz parfait constitué de $N (\sim {\cal N}_A)$ particules relativistes, {\bf indiscernables}, de masse $m$, confinées dans un volume $V$ de dimension 3 : l'énergie cinétique $\epsilon_i$ d'une particule $i$ est donnée par \mbox{$\epsilon_i= \sqrt{m^2c^4+p_i^2c^2}-mc^2$} où  $p_i$ est le module de sa quantité de mouvement $\Vec{p_i}$ et $c$ est la vitesse de la lumière.

On pose $u=\beta mc^2$ et on introduit la variable $x=\beta [\sqrt{m^2c^4+p^2c^2}-mc^2]$.

\question Montrer que la fonction de partition de ce gaz $Z(N,V,T)$ s'exprime en fonction des données et de l'intégrale 
\begin{center}
    $ I(u)=\int_0^{+\infty} dx \ e^{-x}(x+u)\sqrt{(x+u)^2-u^2}$
\end{center}
sous la forme
\begin{center}
    $ Z(N,V,T)=\frac{1}{N!}\left[ \frac{4\pi V}{(\beta h c)^3}I(u) \right]^N$
\end{center}

\question Calculer la pression de ce gaz. Que constate-t-on ?

\question Dans quelle limite obtient-on un gaz ultrarelativiste (comme des photons) ? Calculer dans cette limite l'énergie interne du gaz.

\question Dans quelle limite obtient-on un gaz \og classique \fg ? Calculer dans cette limite l'énergie interne du gaz.

Pour ces deux dernières questions, on donnera dans la limite appropriée en $u$ l'expression équivalente de $I(u)$  mais sans chercher à la calculer.