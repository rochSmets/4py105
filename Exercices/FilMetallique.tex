On considère un fil métallique droit de section constante $s$=1~mm$^2 $ dont le comportement mécanique est analogue à celui d'un ressort~: soumis à une force de traction extérieure $f_{\mathrm{ext}}$, il s'allonge et exerce une force de rappel $f$ sur l'opérateur. Sa longueur $l$ dépend également de sa température $T$ et une équation d'état relie les 3 grandeurs $l = f(T,f)$. Cette équation d'état peut être obtenue à partir de la connaissance de ses coefficients thermoélastiques~: son module d'Young $E$ et son coefficient de dilatation linéaire à force de traction constante $\lambda$  définis de la façon suivante~:
\begin{equation}
E = \dfrac{l}{s} \left( \dfrac{\partial f}{\partial l} \right)_T
\end{equation}  

\begin{equation}
\lambda = \dfrac{1}{l} \left( \dfrac{\partial l}{\partial T} \right)_f
\end{equation}

\question
Sachant que la longueur du fil à la température $T_0$ est $l_0$ lorsqu'il n'est soumis à aucune force de traction ($f$ = 0), montrer que l'équation d'état du fil s'écrit:
\begin{equation}
ln \left( \dfrac{l(T,f)}{l_0} \right) = \dfrac{f}{Es}+\lambda (T-T_0) 
\end{equation} 
\reponse{
A partir de la définition des deux coefficients thermoélastique $E$ et $\lambda$, on obtient deux équations différentielles~:
\begin{equation}
\left( \dfrac{\partial l}{\partial T} \right)_f-\lambda l = 0
\end{equation} 
\begin{equation}
\left( \dfrac{\partial l}{\partial f} \right)_T-\dfrac{l}{Es} = 0
\end{equation}
La résolution de la première équation donne: 
\begin{equation}
\ln l(T,f) = \lambda T+h(f)
\end{equation}
où $h(f)$ représente une fonction de $f$. On dérive cette dernière expression par rapport à $f$ à $T$ constant~:
\begin{equation}
\dfrac{1}{l} \left( \dfrac{\partial l}{\partial f} \right)_T = \dfrac{dh}{df}
\end{equation}
que l'on confronte ensuite à la seconde équation différentielle sur $l$~:
\begin{equation}
\dfrac{dh}{df}=\dfrac{1}{Es}
\end{equation}
que l'on intègre par rapport à $f$~:
\begin{equation}
h(f) = \dfrac{f}{Es}+C
\end{equation}
où $C$ représente une constante qu'on va déterminer à partir de la condition limite $l(T_0, f=0)=l_0$, soit $\ln l(T_0, 0) = \lambda T_0+h(0) = \lambda T_0+C = \ln l_0$.

Finalement, on obtient bien~:
\begin{equation}
\ln l(t,f) = \lambda T + \dfrac{f}{Es} + (\ln l_0-\lambda T_0)
\end{equation}
soit l'équation d'état~: 
\begin{equation}
\ln \left[ \dfrac{l(t,f)}{l_0} \right] = \lambda (T-T_0)+ \dfrac{f}{Es}
\end{equation}
}

\question
En maintenant la condition $l = l_0$, quelle est la relation (algébrique) entre $f$ et $\Delta T$~=~20~K~?
%Si la longueur est fixée à $l_0$, quelle sera la force agissant sur la tige si la température augmente de 20 K ?
(Ceci explique la nécessité de mettre des joints de dilatation dans des systèmes tels que les lignes de chemin de fer, les ponts..)

\donnees{$l_0$ = 5~cm, $E$ = 200~GPa, $\lambda=1,2\times10^{-5}$~K$^{-1}$.}
\reponse{
On a $ l(T_0, f=0) = l_0 = l(T_0+\Delta T,f)$ qui peut s'écrire, en utilisant l'équation d'état~: $\lambda \Delta T = -\dfrac{1}{ES} f$, soit $f = -\lambda ES \Delta T = -48$~N (force de tension interne).
}

\question
Comment se simplifie l'équation d'état pour décrire des états d'équilibre caractérisés par la même température $T_0$ et des valeurs de $f$ faibles devant $Es$? 
\reponse{
D'après l'équation d'état à $T=T_0$, on a $l=l_0 \e ^{f/Es}$. On peut simplifier cette expression pour les petites valeurs de $f/Es$ à l'aide du développement limité en $x=0$, $e^x \sim 1+x$, grâce auquel on obtient $l \simeq l_0 \left( 1+\dfrac{f}{Es} \right)$.
}

\question
On fait subir au fil deux transformations distinctes au cours desquelles il reste en contact avec l'atmosphère. Au cours de la première transformation, on augmente très lentement la force appliquée, de $f_{\mathrm{ext}}=0$ à $f_{\mathrm{ext}}=f_{1} \ll Es$. Au cours de la seconde au contraire, on lui applique $f_{\mathrm{ext}}=f_1$ brutalement. Calculer le travail reçu par le fil dans les 2 cas et comparer les valeurs obtenues.   
\reponse{
Il s'agit de calculer le travail d'une transformation quasi-statique $W_1$ et celui d'une transformation non quasi-statique $W_2$, entre l'état initial A($l_0, T_0, f=0$) et l'état final B($l_1$, $T_0$, $f_1$) où $l_1 = l_0 \e ^{f_1/Es} = l_0(1+f_1/Es)$~:
\begin{equation}
W_1 = \int_{l_i}^{l_f} f_{\mathrm{ext}} \D l = \int_{l_0}^{l_1} f(l)\D l = \int_{l_0}^{l_1} \frac{Es}{l_0}(l-l_0)\D l
\end{equation}
qui vaut $W_1 = \frac{Es}{2l_0}(l_1-l_0)^2=f_1^2 \frac{l_0}{2Es}$.
\begin{equation}
W_2 = \int_{l_i}^{l_f} f_{\mathrm{ext}}\D l = f_1 \int_{l_0}^{l_1}\D l = f_1 (l_1-l_0) 
\end{equation}
qui vaut $W_2 = \frac{Es}{l_0}(l_1-l_0)^2 = f_1^2\frac{l_0}{Es} = 2W_1 $.
}
