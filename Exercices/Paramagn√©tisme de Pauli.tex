On considère un gaz d'électrons libres dans une enceinte de volume $V$ à la température $T$.

\question
Rappeler l'expression de la densité d'états en énergie $\rho_{3D}(\epsilon)$.

\question
Rappeler quel est le nombre moyen d'occupation $n(\epsilon)$ d'un état quantique d'énergie $\epsilon$ lorsque le potentiel chimique du gaz est égal à $\mu$.

\question
Rappeler l'expression de l'énergie de Fermi à température nulle.

\medskip

On plonge le gaz d'électrons dans un champ magnétique $\vec{B}=B_z\vec{e_z}$. Soit $\vec{m}$ le moment magnétique d'un électron.

\question
Montrer que la densité d'états en énergie des électrons up (moment aligné avec le champ) et down (dans le sens contraire) est
$$
\rho_\pm(\epsilon)=\frac{1}{2}\rho_{3D}(\epsilon \pm mB), \epsilon \geq \mp mB
$$

\question
Calculer le nombre $N_\pm$ d'électrons up et down à température nulle et écrire la relation qui relie le niveau de Fermi $\epsilon_F$ à $N, V$ et $B$.

\question
La valeur de $m$ est égale au  magnéton de Bohr soit environ 10$^{-23}$ J.T$^{-1}$. Montrer que $mB \ll \epsilon_F(B=0)$ et que, au premier ordre, $\epsilon_F(T=0,B)$ est indépendant de $B$.

\question
En déduire que l'aimantation à température nulle se met sous la forme $M=\frac{3}{2} N m \frac{mB}{\epsilon_F}$. Comparer avec le résultat équivalent pour des particules discernables, expliquer en particulier l'origine de l'énorme réduction de l'aimantation.

\question
\'Ecrire $N_\pm$ et $M$ pour des températures non nulles. Discuter du comportement de $M$ lorsque $T\gg T_F$. Montrer que dans cette limite, le potentiel chimique est donné par

$$
\exp (\beta \mu)=\frac{4(\beta \epsilon_F)^{\frac{3}{2}}}{3\sqrt{\pi}\cosh(\beta m B)}. \nonumber
$$
et que l'aimantation vérifier l'équation $M= Nm\tanh(\beta m B)$ comme attendu.

