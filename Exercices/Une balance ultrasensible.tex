On accroche une masse $m$ à une balance très sensible constituée par un ressort sans masse de raideur $\alpha$. L'ensemble est à l'équilibre thermique avec une enceinte à la température $T$. On désigne par $g$ l'accélération de la pesanteur et on introduit $\omega^2=\frac{\alpha}{m}$. On suppose que le seul degré de liberté du système masse-ressort est la position selon la verticale de la masse. L'hamiltonien  s'écrit alors ${\cal H}=\frac{p_z^2}{2m}+mgz+\frac{1}{2} m\omega^2 z^2$.

\medskip

\question Donner le sens des différents termes de l'hamiltonien. \`A quel choix de l'origine ($z=0$) correspond cet hamiltonien ? 

\question Calculer la fonction de partition $Z$ du système. Sous quelles conditions peut-on considérer que $-\infty < z < +\infty$ ? On rappelle que $\int_{-\infty}^{+\infty} \exp(-u^2) du= \sqrt{\pi}$. Montrer par un changement de variable approprié que $Z=\frac{1}{\beta \hbar \omega} \exp(\frac{\beta m g ^2}{2 \omega^2})$

\question Relier la valeur moyenne $\overline z$ de la position à une dérivée partielle de $\ln Z$ par rapport à une variable judicieusement choisie. Faire de même avec $\sigma^2=\overline{z^2} - \overline{z}^2$   et une dérivée partielle seconde de  $\ln Z$.

\question Calculer $\overline z$. Ce résultat était-il prévisible ? Calculer $\sigma^2$. Quelle est l'origine physique des fluctuations de la position de la masse ?

\question Pensez vous que l'on puisse construire un ressort permettant de mesurer la constante de Boltzmann à partir de la mesure des fluctuations de position ? On justifiera sa réponse en donnant des ordres de grandeurs.
