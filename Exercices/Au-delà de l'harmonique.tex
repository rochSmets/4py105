On considère un oscillateur harmonique (OH) classique à une dimension en équilibre avec un thermostat à la température $T$. Il est modélisé par une masse $m$ plongé dans un potentiel harmonique. L'énergie de l'oscillateur prend alors la forme ${\cal H}(x,p)=\frac{K}{2}x^2+\frac{p^2}{2m}$ où $x$ est l'écart à la position d'équilibre, $p$ la quantité de mouvement associée et $K$ la raideur du potentiel harmonique.

On rappelle les intégrales de Gauss suivantes:

\begin{center}
$I_0(\alpha)=\int_{-\infty}^{+\infty} dx \ e^{-\alpha x^2}=\sqrt{\frac{\pi}{\alpha}}; I_2(\alpha)=\int_{-\infty}^{+\infty} dx \ x^2 e^{-\alpha x^2}=\frac{1}{2\alpha}\sqrt{\frac{\pi}{\alpha}};I_4(\alpha)=\int_{-\infty}^{+\infty} dx \ x^4 e^{-\alpha x^2}=\frac{3}{4\alpha^2}\sqrt{\frac{\pi}{\alpha}}.$    
\end{center}

\question Quelle est l'énergie moyenne $u$ de cet OH ? En déduire sa capacité calorifique.

\question Quelle est la densité de probabilité $P(x)$ de trouver l'OH dans un écart  $x$ ? ($P(x)dx$ est la probabilité de trouver l'OH dans un écart à l'équilibre situé dans l'intervalle $[x,x+dx]$).

\question En déduire $\langle x \rangle$ la valeur moyenne de $x$ ainsi que $\langle x^2 \rangle$. 

\

Un opérateur modifie par une transformation réversible la raideur du potentiel harmonique de la valeur $K$ à la valeur $K'$. Considérons la transformation infinitésimale de $K$ à $K+\delta K$.

\

\question De combien  a varié l'énergie (respectivement la densité de probabilité) de l'OH dans un écart $x$ à l'ordre le plus bas en $\delta K$ ? On notera resp. $\delta E(x)$ et $\delta P(x)$ ces variations. NB: il sera utile de les écrire le plus possible en fonction de $\frac{\delta K}{K}$, $E(x)=\frac{Kx^2}{2}$ et $P(x)$.

\question Soit $\delta T=\int_{-\infty}^{+\infty} \D x \ P(x) \delta E(x)$. Quel est le sens de $\delta T$ ? Calculer $\delta T$.

\question Soit $\delta C=\int_{-\infty}^{+\infty} \D x \ E(x) \delta P(x)$. Quel est le sens de $\delta C$ ? Calculer $\delta C$.

\question Que constate-t-on ? Aurait-on pu l'anticiper à partir de la thermodynamique ?

\question Calculer le travail reçu par l'OH quand on passe la raideur de $K$ à $K'$.

\question Calculer la variation d'entropie de l'OH au cours de cette transformation.

\

Le potentiel harmonique n'est en général qu'une approximation des potentiels réels. On s'intéresse donc désormais au potentiel $V(x)=\frac{K}{2}x^2-g x^3-f x^4$. Les deux derniers termes sont des corrections par rapport au potentiel harmonique si bien que $g\sqrt{\langle x^2 \rangle}^3 \ll k_B T$ et $f\sqrt{\langle x^2 \rangle}^4 \ll k_B T$ où $\langle x^2 \rangle$ est la valeur calculée auparavant pour le potentiel harmonique. 

\

\question Calculer la fonction de partition $Z$ de cet oscillateur anharmonique à l'ordre le plus bas en $g$ et $f$. Montrer que $Z \simeq \frac{2\pi}{h \beta} \sqrt{\frac{m}{K}} \left( 1+\frac{3f}{\beta K^2} \right)$.

\question En déduire la nouvelle énergie moyenne $u^{\prime}$ de l'oscillateur (toujours à l'ordre le plus bas en $g$ et $f$) et sa nouvelle capacité calorifique $c^{\prime}$. Que constate-t-on ?

\question Calculer la nouvelle valeur moyenne de $x$ à l'ordre le plus bas en $g$ et $f$. Que constate-t-on ?

\question Dans ce qui précède, quel est selon vous le terme important pour expliquer la dilatation des solides avec la température (que l'on n'a pas avec le modèle d'Einstein par exemple) ?


