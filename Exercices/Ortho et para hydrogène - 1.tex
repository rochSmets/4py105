Dans son état électronique fondamental, la molécule d'hydrogène $H_2$ peut exister sous deux formes : l'ortho-hydrogène où les spins des deux noyaux sont parallèles et le para-hydrogène, où ils sont antiparallèles. La forme para possède un seul état de spin dont on prendra l'énergie comme origine; la forme ortho présente trois états distincts, de même énergie $\epsilon >0$. On considère un échantillon d'hydrogène solide, constitué de $N$ molécules, fixes et discernables, faiblement couplées. On ne s'intéresse qu'aux états de spin ortho et para. Le système est isolé et son énergie (de spins) vaut $E (\gg \epsilon)$.

\question
Sur quel domaine peut-varier l'énergie $E$ ?

\question
Calculer le nombre $\Omega(E)$ de micro-états d'énergie $E = n \epsilon$.

\question
Dans l'hypothèse où $N$ et $n$ sont très grands, calculer l'entropie $S(E,N)$ du système.

\question
En déduire la température $T$ en fonction de $E, N, \epsilon$ et $k_B$.

\medskip

\`A l'équilibre thermodynamique, cette température de spins est égale à la température du matériau.

\question
\`A température ambiante, on mesure environ une répartition de 70\% d'ortho-H$_2$ et 30\% de para-H$_2$. En déduire un ordre de grandeur de l'écart énergétique $\epsilon$ entre les formes para et ortho.

\question
On lit souvent que \og l'ortho-hydrogène est instable à basse température, et se transforme spontanément en para-hydrogène avec le temps, ce qui libère de la chaleur indésirable. \fg \ Comprenez-vous cette assertion ? Estimer à 20 K la fraction ortho/para.
