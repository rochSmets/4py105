Einstein fut le premier à proposer en 1907 un modèle capable de rendre compte du comportement des capacités calorifiques des solides à basse température : leur diminution sensible, en contradiction avec la loi classique de Dulong et Petit.

Voici le modèle : on admet que chaque atome du solide vibre indépendamment des autres atomes à la même pulsation $\omega $ dans chacune des trois directions possibles. Le solide, formé de $N$ atomes est donc équivalent à un ensemble de
$3N$ oscillateurs à une dimension indépendants de pulsation propre $\omega$. On sait que les états propres de l'Hamiltonien de chacun de ces oscillateurs ont des niveaux discrets non dégénérés d'énergie $\epsilon_n$ caractérisée par un nombre quantique $n$ ($n=0,1,2,3\ldots$) telle que
$$
  \epsilon_n=(n+\frac{1}{2}) \hbar \omega.
$$

Dans tout ce qui suit, on suppose que le solide est en équilibre thermique avec un thermostat à la température $T$. On pourra introduire la grandeur $\theta=\hbar \omega/k_B$ que l'on interprétera.

\question
Rappeler le lien entre la fonction de partition d'un système en équilibre thermique avec son énergie libre, son énergie interne (i.e. son énergie moyenne), puis sa capacité calorifique à volume constant.

\question Calculer la fonction de partition $z$ d'un oscillateur harmonique quantique à une dimension (pour calculer la somme on remarquera qu'il s'agit d'une simple série géométrique).

\question Calculer l'énergie moyenne $\overline \epsilon (T)$ de cet oscillateur. Tracer qualitativement $\overline \epsilon (T)$ en fonction de $T$. Quelle est l'énergie interne $\overline E$ du solide dans le cas du modèle d'Einstein ?

\question Dans la limite où $T \ll \theta$, sans aucun calcul, que peut-on dire de la valeur de $\overline E$ ? Vérifier-le sur la formule précédente.

\question Dans la limite où $T \gg \theta$, que vaut $\overline E$ ? Comment dépend-elle de $T$ ? de $\omega $ ? 

\question Montrer que dans le modèle d'Einstein, la capacité calorifique molaire à volume constant $C_V$ du solide est égale à
$$
 C_V=3R \frac{x^2 \exp ( x)}{[\exp (x)-1]^2}
$$
ou $R$ est la constante des gaz parfaits et $\displaystyle{x=\frac{\theta}{T}}$.

\question
Tracer sommairement $C_V(T)$en fonction de $T$. \'Etudier en particulier les limites $C_V(T \to + \infty)$, puis $C_V(T \to 0)$. Donner une expression approchée de $C_V(T)$ quand $T \ll \theta$.

\question
On donne à 298 K pour le cuivre $\theta=230$K et $C_V=23,8$ J.K$^{-1}$.mol$^{-1}$ tandis que pour le diamant $\theta=830$ K et $C_V=6.1$ J.K$^{-1}$.mol$^{-1}$. Commenter.
