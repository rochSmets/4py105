On étudie un gaz parfait constitué de $N$ bosons indépendants de masse $m$ et de spin nul, libres mais astreints à se déplacer sur une surface d'aire $S$ et à l'équilibre à la température $T$.

\question
Déterminer la densité d'état en énergie $\rho_{2D}(\epsilon)$ (en appliquant des conditions aux limites périodiques pour le vecteur d'onde). 

\question
Rappeler quel est le nombre moyen d'occupation $n(\epsilon)$ d'un état quantique d'énergie $\epsilon$ lorsque le potentiel chimique du gaz est égal à $\mu$. Pourquoi a-t-on $\mu<0$ ?

\question
En déduire la relation implicite qui détermine le potentiel chimique $\mu$ en fonction de $N$, $S$ et $T$. Dans toute la suite, on notera $f=e^{\beta \mu}$

\question
Montrer que $\frac{N}{S}\Lambda^2=-\ln (1-f)$ où $\Lambda$ est une longueur que l'on exprimera en fonction des données.

\question
Vérifier que la formule précédente vous redonne bien, dans la limite classique (non quantique)
l'expression de la fugacité d'un gaz parfait classique.

\question
Conclure quant à l'existence, ou l'absence d'un phénomène de condensation de Bose-Einstein en deux dimensions.

\medskip

On piège à présent notre assemblée d'atomes par un potentiel harmonique magnétique $V(r)=\frac{1}{2}m\omega^2 r^2$. Les états propres de chaque atome sont indexés par deux entiers $i, j \geq 0$, et leur énergie associée est $\epsilon_{i,j} = \hbar \omega (i + j)$. 

\question
Dans le contexte expérimental qui nous concerne (des atomes de rubidium 87), on a  $T \sim 100$ nK et $\omega \sim 2\pi \times 10$ Hz. Estimer numériquement $\beta \hbar \omega$.

\question
Exprimer, en fonction de $n$, le nombre $g_n$ d'états accessibles à un atome
occupant le niveau d'énergie $n \hbar \omega$.

\question
Exprimer $N_0$ le nombre d'atomes occupant l'état fondamental, en fonction de $f$.

\question
Exprimer $N_e$ le nombre d'atomes occupant des états excités sous la forme d'une somme sur des entiers $\geq 1$, d'une fonction de $\frac{N}{S}, \beta \hbar \omega$ et $f$. 

\question
Compte-tenu de l'application numérique ci-dessous, on admet que l'on peut approcher $N_e$ par une intégrale. On introduit la fonction
$$
g_2(f)=\frac{1}{\Gamma(2)} \int_0^{+\infty} dx \frac{x}{\frac{e^x}{f}-1} = \sum_{k\geq 1} \frac{f^k}{k^2}
$$
Exprimer $N_e$ en fonction de $g_2(f)$ et des données.

\question
$g_2$ est une fonction croissante. Quel est le maximum possible pour $N_e$ en fonction de la température ?  On donne $\sum_1^{+\infty} \frac{1}{k^2}=\frac{\pi^2}{6}$.

\question
Conclure quant à l'existence, ou l'absence, d'un phénomène de condensation de Bose-Einstein en deux dimensions en présence d'un piège harmonique.

