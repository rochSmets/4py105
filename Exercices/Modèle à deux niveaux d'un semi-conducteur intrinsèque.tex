On se propose d'étudier la conductivité électrique d'un semi-conducteur intrinsèque, c'est-à -dire ne contenant pas d'impuretés. On considère donc un modèle dans lequel les bandes de valence et de conduction sont assimilées à  deux niveaux d'énergie $\epsilon_1$ et $\epsilon_2$, de même dégénérescence $g=g_1=g_2$. On posera $\Delta \epsilon=\epsilon_2-\epsilon_1 > 0$. Soit $N$ le nombre d'électrons se trouvant dans la bande de valence (c'est-à -dire dans ce modèle sur $\epsilon_1$) au zéro absolu:  à  cette température, la bande de conduction est vide. On suppose donc que $N=g$. Soient $T$ la température du système et $\mu$ son potentiel chimique.

\question{A quelle statistique obéissent les électrons ? En déduire le nombre d'électrons $N_1$ et $N_2$ se trouvant respectivement sur les niveaux $\epsilon_1$ et $\epsilon_2$ à  l'équilibre thermique. Quelle relation lie $N_1$, $N_2$ et $N$ ? }

\question{On pose $f=\exp(\beta \mu)$, $f_1=\exp(\beta \epsilon_1)$ et $f_2=\exp(\beta \epsilon_2)$.  Déduire de la relation précédente l'expression de $f$ en fonction de $f_1$ et $f_2$. Montrer que, quelle que soit la température, $\mu=\frac{\epsilon_1+\epsilon_2}{2}$.}

\question{On note  $P_1$ et $P_2$, le nombre de trous ou places vides dans chacun des niveaux 1 et 2. Exprimer $N_1$ et $N_2$, ainsi que $P_1$ et $P_2$ en fonction de $g, \Delta \epsilon$ et $k_B T$. Quelles sont les limites de ces quantités aux basses et aux hautes températures ?}

\question{Comparer ces résultats avec ceux que l'on obtient lorsqu'on applique la statistique classique de Maxwell-Boltzmann à  ce système.}

\question{Donner les expressions de l'énergie moyenne $\bar{E}$ et de la capacité calorifique $C$ en fonction de la température.}

\question{On admet que $C$ présente un maximum lorsque $2\frac{k_BT}{\Delta \epsilon}=0,42$. Sachant que $\Delta \epsilon=1$ eV, calculer la température correspondante. En déduire que dans les semi-conducteurs usuels, seule la partie croissante de la capacité calorifique avec la température peut être observée.}

\question{Toujours dans le cas $\Delta \epsilon=1$ eV, simplifier les expressions obtenues de $N_2$ et $P_1$ lorsque la température est proche de l'ambiante (300 K).}

\question{La conductivité électrique est donnée par la formule

\begin{align*}
\sigma=e\left(\frac{N_2}{V} \mu_-+\frac{P_1}{V} \mu_+\right),
\end{align*}
où $e$ est la charge élémentaire, $\mu_\pm$ les mobilités respectives des électrons et des trous dans ce cristal, quantités que l'on supposera indépendantes de la température. Pour le silicium, $n=\frac{N}{V}= 3,1.10^{19}$ cm$^{-3}$, $\Delta \epsilon=1,1$ eV, $\mu_+=400$ cm$^2$.(V.s)$^{-1}$ et $\mu_-=1600$ cm$^2$.(V.s)$^{-1}$. Calculer la conductivité électrique du silicium pur à  $T=300$ K et $T=1000$ K. Tracer la courbe théorique de la conductibilité en fonction de la température dans une représentation où le graphe de la fonction est linéaire. Expliquer comment la mesure de $\sigma$ permet de déterminer $\Delta \epsilon$. }

