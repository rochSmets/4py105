
On considère un gaz parfait constitué de $N (\sim {\cal N}_A)$ particules relativistes, \important{indiscernables}, confinées dans un volume $V$ de dimension 3 : l'énergie  $\epsilon_i$ d'une particule $i$ de masse $m$ est donnée par $\epsilon_i= \sqrt{p_i^2c^2+m^2c^4}$ où $c$ est la vitesse de la lumière et $p_i=|\Vec{p_i}|=\hbar |\Vec{k_i}|$ le module de sa quantité de mouvement $\Vec{p_i}$ (appelée aussi impulsion) associée au vecteur d'onde $\Vec{k_i}$.  

Ce gaz est constitué de fermions de spins $\frac{1}{2}$, à la température $T$ et au potentiel chimique $\mu$. On admet que la densité d'état dans l'espace réciproque (des vecteurs d'onde) est $\hat{\rho}(\Vec{k})d^3\Vec{k}=2 \frac{V}{(2 \pi)^3} d^3 \Vec{k}$.

\question Montrer que la densité d'état en module de l'impulsion $\rho(p)$ est de la forme $A p^2$ où $A$ s'exprime en fonction de $V$ et $\hbar$. 

\question Rappeler l'expression du nombre moyens de particules $n_i^F(\epsilon)$ dans un état quantique $i$ d'énergie $\epsilon$ en fonction de $\beta, \epsilon$ et $\mu$. En déduire la même grandeur dans un état quantique $i$ d'impulsion $\Vec{p}$.

\question \'Ecrire les expressions formelles, dans l'approximation continue,   du nombre moyen de particules $N$ et de l'énergie moyenne $U$ sous forme d'intégrales sur le module $p$ de l'impulsion.

\ 

On fait l'hypothèse que le fluide de fermions est dégénéré c'est-à-dire que la température peut être considérée comme nulle.

\question Que devient $n_i^F(\epsilon)$ dans cette limite ? On introduit $p_F$, appelée impulsion de Fermi, l'impulsion qui vérifie $\mu=\sqrt{p_F^2c^2+m^2c^4}$.  En déduire la relation suivante entre $N$, $V$ et  $p_F$: $N= \frac{V}{\pi^2 \hbar^3}\int_0^{p_F} \D p \ p^2 $. Montrer que $p_F=\hbar \sqrt[3]{3 \pi^2\rho}$ où $\rho=\frac{N}{V}$.

\question Montrer que l'énergie interne $U$ du gaz de Fermion s'exprime comme $U=V\frac{m^4c^5}{\pi^2 \hbar^3}\int_0^{x_F} \D x \ x^2 \sqrt{1+x^2}$, où $x_F=\frac{p_F}{mc}$.

\question Que peut-on dire de $p_F$ dans la limite non relativiste ? En déduire que $U=V\frac{m^4c^5}{\pi^2 \hbar^3}(\frac{x_F^3}{3}+\frac{x_F^5}{10}+\ldots)$. Exprimer $U$ en fonction de $N$ et interpréter les deux premiers termes de ce développement. 

\question Que peut-on dire de $p_F$ dans la limite ultra-relativiste ? En déduire avec audace ou admettre avec modestie que $U=V\frac{m^4c^5}{\pi^2 \hbar^3}(\frac{x_F^4}{4}+\frac{x_F^2}{4}+\ldots)$. Cette formule vous servira ci-dessous.



\ 

Une naine blanche est une étoile (sphérique) de masse comparable à celle du soleil mais de densité beaucoup plus forte de sorte que ses électrons constituent un gaz de Fermi dégénéré. Pour fixer les idées, on supposera que son rayon est $R= 5 000$ km, sa masse $M=2\times 10^{30}$ kg, et qu'elle est constituée principalement d'atomes de carbone $^{12}C$ ($Z=6$ protons, $A=12$ nucléons) totalement ionisés avec des températures de coeur de l'ordre de $T \approx 10^7$ K.

\question Calculer numériquement le nombre d'atomes de carbone dans l'étoile. On donne $m_P \cong m_N \cong 1,67 \times 10^{-27}$ kg. En déduire le nombre d'atomes de carbone par unité de volume en supposant cette grandeur homogène au sein de l'étoile, puis le nombre d'électrons $\rho$ par unité de volume. On écrira aussi $\rho$ de façon littérale en fonction de $M, R, m_P, Z$ et $A$.

\question Calculer l'impulsion de Fermi des électrons dans la naine blanche, puis l'énergie cinétique des électrons au niveau de Fermi ($\sqrt{p_F^2c^2+m^2c^4}-mc^2)$ en MeV. Justifier que les électrons peuvent être considérés comme très relativistes. On donne pour l'électron $mc^2=0,51$ MeV.


On admet que l'énergie $E$ de l'étoile est égale à la somme de l'énergie des électrons, considéré comme un gaz de Fermi dégénéré et ultra-relativiste et de l'énergie gravitationnelle
de l'étoile $E_G=-\frac{3}{5}\frac{GM^2}{R}$ où $G$ est la constante gravitationnelle $G=6,67 \times 10^{-11}$  m$^3$.kg$^{-1}$.s$^{-2}$. 

\question Montrer que l'énergie de l'étoile s'exprime approximativement sous la forme $E=-\alpha \frac{M^2}{R}+\delta \frac{M^{\frac{4}{3}}}{R}+\gamma M^{\frac{2}{3}} R+\ldots$. Donner l'expression des coefficients $\alpha, \delta$ et $\gamma$ en fonction de constantes et du rapport $\frac{A}{Z}=2$. Il sera judicieux de commencer par écrire $x_F$ en fonction de $M, R$ et de constantes.

\question Donner l'équation qui détermine la valeur du rayon $R_m$ qui rend minimale l'énergie de l'étoile. Montrer que cette équation n'a de solution que si la masse de l'étoile est inférieure à une masse $M_C$ que l'on déterminera en fonction de $\alpha$ et $\delta$. 

\question Calculer numériquement $M_C$ et $R_m$ (good luck !).

\question Que se passe-t-il selon vous si $M > M_C$ ?

