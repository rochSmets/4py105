Lorsqu'il y a empoisonnement par le monoxyde de carbone, les molécules de $CO$ remplacent les molécules d'$O_2$, adsorbées sur les molécules d'hémoglobine ($Hb$) du sang. Pour montrer cet effet, on considère un modèle dans lequel chaque site des molécules $Hb$ peut être vacant avec une énergie nulle, ou occupé, soit par une molécule $O_2$ avec l'énergie $E_A$, soit par une molécule de $CO$ avec une énergie $E_B$. L'occupation d'un site n'a aucune influence sur l'occupation des autres sites.

On considère $M$ molécules (discernables) d'$Hb$ avec un site d'adsorption disponible chacune, en équilibre à la température du corps humain $T=37^{\circ} $ C avec de l'oxygène $O_2$ et du monoxyde de carbone $CO$ en phase gazeuse, à des densités moléculaires telles que les activités absolues soient $\lambda_{O_2}=\exp(\beta \mu_{A})=10^{-5}$ et $\lambda_{CO}=\exp(\beta \mu_{B})=10^{-7}$.

\medskip


\question Quels sont les états d'occupation et les énergies correspondantes pour chaque site d'adsorption ? En déduire que la grande fonction de partition $\zeta(T, \mu_A,\mu_B)$ associée à un site s'écrit $\zeta=1+\lambda_{O_2} e^{-\beta E_A} +\lambda_{CO} e^{-\beta E_B}$.

\question Justifier que la grande fonction de partition $\Xi$ des $M$ sites s'écrive $\Xi=\zeta^M$.

\question Calculer $\phi_A$ et $\phi_B$, le pourcentage moyen d'occupation  des sites respectivement par l'oxygène $O_2$ et par le monoxyde de carbone $CO$.

\question On se place dans une situation en l'absence de $CO$ ($\lambda_{CO}=0$). On constate que $90\%$ des sites $Hb$ sont occupés par une molécule de $O_2$. Déterminer $E_A$ en électron-volts.

\question On est en présence des deux gaz dans les conditions précisées au début. On constate qu'il n'y a plus que $10\%$ des sites $Hb$ occupés par une molécule de $O_2$. Déterminer $E_B$ en électron-volts.

\question Rappeler l'expression du potentiel chimique d'un gaz parfait en fonction de la température $T$, de sa densité moléculaire $n$ et de la longueur thermique de De Broglie $\Lambda$ associée. En assimilant $O_2$ et $CO$ à des gaz parfaits, à quel rapport $\frac{n_A}{n_B}$ correspond les activités données au début ? Commenter le résultat.
