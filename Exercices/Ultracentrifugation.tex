Un cylindre de révolution de rayon $R$, de hauteur $h$, d'axe $Oz$ contient $N$ molécules de masse $m$ d'un
gaz parfait en équilibre à la température $T$. On fait tourner ce cylindre autour de son axe
à la vitesse angulaire constante $\omega$. En régime permanent (lorsque le gaz a  été entraîné à la vitesse $\omega$ autour de $Oz$),  on admet qu'une molécule située à une distance $r$ de l'axe de rotation est soumise à une force centrifuge égale en module à $m\omega^2 r$.

\question
Comparer la force centrifuge au poids (à la surface de la Terre)  pour un rayon $r=20$ cm et pour une vitesse de rotation de 10 000 tours par minute.

\question
De quelle énergie potentielle $E_p(r)$ dérive cette force (attention au signe !) ?

\question
Calculer le quotient $\frac{p(r)}{p(0)}$ entre la probabilité $p(r)$ pour une particule d'être à la distance $r$ de l'axe et la probabilité $p(0)$ pour une particule d'être sur l'axe. Expliquer pourquoi ce quotient est aussi égal au rapport des densités particulaires  $\frac{n(r)}{n(0)}$. Application numérique avec les valeurs précédentes pour un gaz de masse molaire 352 g (de l'hexafluorure d'uranium) et à température ambiante. Commenter.

\question
Calculer l'expression de l'intégrale de configuration $Q_c$ associée à cette énergie, puis de l'énergie moyenne correspondante.
