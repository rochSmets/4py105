Un système isolé est constitué de $N$ électrons piégés sur les $N$ sites d'un solide cristallin. Chaque site peut accueillir au plus deux électrons et on a donc trois états possibles pour chacun des sites~:

\begin{itemize}
\item aucun électron n'est piégé: on prend cet état comme zéro d'énergie.
\item un électron est piégé: l'énergie de cet état est alors $-\epsilon<0$.
\item deux électrons sont piégés: l'énergie de cet état est $-2\epsilon+g$ où $g>0$. 
\end{itemize}

On note $n_0, n_1$ et $n_2$ le nombre de sites occupés respectivement par 0, 1 ou 2 électrons.

\question
Selon vous, pourquoi ne peut-on pas mettre plus de deux électrons par site et pourquoi $g>0$ ?

\question
Exprimer le nombre $N$ de sites du solide en fonction de $n_0, n_1$ et $n_2$. Faire de même pour le nombre $N$ d'électrons en fonction de $n_1$ et $n_2$. Montrer que $n_0=n_2$.

\question
Exprimer l'énergie $E$ en fonction de $\epsilon, g, n_1$ et $n_2$. Entre quelles limites peut-on choisir la valeur de $E$ ?

\question
Déterminer $n_0, n_1$ et $n_2$ en fonction de $E, N, \epsilon$ et $g$.

\question
Justifier que le nombre $\Omega(E,N)$ de micro-états d'énergie $E$ soit égal à  $\Omega(E,N)=\frac{N!}{n_0!n_1!n_2!}$.

\question{En déduire l'entropie $S(E,N)$ du système dans la limite thermodynamique (où tous les nombres sont de l'ordre du nombre d'Avogadro) en fonction de $N,n_1$ et $n_2$. On utilisera la formule de Stirling que l'on rappellera au préalable.}

\question
Calculer la température micro-canonique $T$ du système et montrer qu'elle peut s'exprimer sous la forme $\frac{1}{T}=\frac{2k_B}{g}\ln(\frac{n_1}{n_2})$.

\question
En déduire l'expression de $E$ en fonction de $N, T, \epsilon$ et $g$. Quelle est la limite de $E$ lorsque $k_B T \gg g$ ? Comment interprétez vous le résultat lorsque $g=0$ ?

\question
Calculer $n_0, n_1$ et $n_2$ en fonction de $N$,  $T, \epsilon$ et $g$. Commenter les limites de ces nombres lorsque $k_B T \ll g$ et lorsque $k_B T \gg g$.
