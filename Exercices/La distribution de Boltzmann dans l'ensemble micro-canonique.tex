On rappelle que $N$ atomes de masse $m$ d'un gaz parfait monoatomique isolé d'énergie $E$ dans un volume $V$, le nombre de micro-états $\Phi(E,V,N)$  d'énergie inférieure à $E$ est de la forme
$$
\Phi(E, V, N)= \frac{V^N}{N!}\frac{(\frac{2\pi m E}{h^2})^{\frac{3N}{2}}}{(\frac{3N}{2})!}
$$

\question
Calculer l'entropie $S(E,V,N)$ du gaz et vérifier qu'elle est extensive. Expliquer pourquoi l'extensivité nécessite de tenir compte de l'indiscernabilité des atomes.

\question
Calculer la température micro-canonique $T$ du système et exprimer $E$ en fonction de $N$ et de $T$. Quel résultat bien connu retrouve-t-on ?

On isole par la pensée une particule. 

\question
Montrer que la probabilité que cette particule soit située en $\vec r$ à $d\vec r$ près avec la quantité de mouvement $\vec p$ à $d\vec p$ près s'exprime comme
$$
P(\vec r,\vec p)d\vec r d\vec p=\frac{d\vec r d\vec p}{N h^3} e^{\frac{1}{k_B}\left[S(E-\frac{p^2}{2m},V,N-1)-S(E,V,N) \right]}
$$

\question
Développer $S(E-\frac{p^2}{2m},V,N-1)-S(E,V,N)$ au premier ordre et en déduire que la probabilité que la particule ait la quantité de mouvement $\vec p$ à $d\vec p$ est proportionnelle à $e^{-\frac{p^2}{2mk_BT}}$. Quel résultat bien connu retrouve-t-on ?
