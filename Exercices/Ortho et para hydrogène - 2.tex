Dans son état électronique fondamental, la molécule d'hydrogène H$_2$ peut exister sous deux formes : l'ortho-hydrogène où les spins des deux noyaux sont parallèles et le para-hydrogène, où ils sont antiparallèles. La forme para possède un seul état de spin dont on prendra l'énergie comme origine; la forme ortho présente trois états distincts, de même énergie $\epsilon >0$. On considère un échantillon d'hydrogène solide, constitué de $N$ molécules, fixes et discernables, faiblement couplées. On ne s'intéresse qu'aux états de spin ortho et para. Ce cristal est en contact avec un thermostat à la
température $T$

%\elements{
%\\
%1 : $Z=z^N$ avec pour une molécule $z=\sum_{i=1}^4 {\rm e}^{-\beta \epsilon_i}=1+3{\rm e}^{-\beta \epsilon}$.
%\\
% 2 : $\langle E\rangle=-\frac{\partial \ln Z}{\partial \beta}=N \epsilon \frac{3{\rm e}^{-\beta \epsilon}}{1+3{\rm e}^{-\beta \epsilon}}$,  $C=\frac{\partial \langle E \rangle}{\partial T}= 3Nk (\beta \epsilon)^2 \frac{{\rm e}^{\beta \epsilon}}{(3+{\rm e}^{\beta \epsilon})^2}$ et $<n_{para}>=\frac{N}{z}=\frac{N}{1+3{\rm e}^{-\beta \epsilon}} $ et $<n_{ortho}>=\frac{3N{\rm e}^{-\beta \epsilon}}{z}=\frac{3N{\rm e}^{-\beta \epsilon}}{1+3{\rm e}^{-\beta \epsilon}}$.
%\\
%3 : $S= \frac{1}{T}(E-F)$, avec $F=-kT \ln Z$, donc $S=kN \Big[\frac{3 \beta \epsilon }{3+{\rm e}^{\beta \epsilon}} +\ln (1+3{\rm e}^{-\beta \epsilon})\Big] \to kN \ln 4$ quand $\beta \epsilon \ll 1$, car il y a $4^N$ micro-états équiprobables à haute température (et $S \to 0$ quand $\beta \epsilon \gg 1$).
%}

\question
Calculer la fonction de partition $Z$ du cristal.

\question
En déduire l'énergie moyenne $\langle E \rangle$, la capacité calorifique $C$ et les valeurs moyennes du nombre de molécules dans les états para et ortho.

\question
Exprimer et tracer l'entropie $S$ en fonction de la température.  Quelle est la limite de $S(T)$ à haute température, quand $k_BT \gg \epsilon $, puis à base température, quand $k_B T \ll \epsilon$ ?
