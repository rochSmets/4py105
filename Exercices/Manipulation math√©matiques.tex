{\sffamily\bfseries{L'intégrale Gaussienne}}

Soit l'intégrale gaussienne $ \displaystyle I_n(\alpha) = \int_{0}^{\infty} x^{n} {\rm e}^{-\alpha x^2} {\rm d}x\enspace, \textrm{ où } \alpha>0 $.

\question
Montrer que $I_1(\alpha)=\frac{1}{2\alpha}$.

\question
Exprimer $I_n(\alpha)$ en fonction de $I_{n-2}(\alpha)$.

\question
On admet en sus que $I_0(\alpha)=\frac{1}{2}\sqrt{\frac{\pi}{\alpha}}$. En déduire que $I_2(\alpha)=\frac{1}{4\alpha} \sqrt{\frac{\pi}{\alpha}}$ et que $I_3(\alpha)= \frac{1}{2\alpha^2}$.

\bigskip

{\sffamily\bfseries{La fonction  Gamma et la factorielle}}

On définit la fonction Gamma,  
$
\displaystyle \Gamma(x)  = \int^\infty_0 {\rm e}^{-t} t^{x-1} {\rm d}t 
$, pour $x>0$.

\question
Calculer $\Gamma(1)$ et $\Gamma (1/2)$.

\question
Montrer que $ \Gamma (x+1) = x \Gamma(x)$. En déduire $\Gamma(N+1)$ en fonction de $N!$, où $N$ est un entier positif.

\bigskip

{\sffamily\bfseries{Les factorielles des grands nombres et la formule de Stirling}}

On montre que la factorielle $n!$ d'un nombre $n$ entier peut s'approcher par le développement suivant : 
$$
\label{eqStirling}
n! = \sqrt{2 \pi n} \; n^n \; e^{-n} \; \Big[ 1 + \frac{1}{12\, n} + \frac{1}{288\, n^2}+ O(\frac{1}{n^3}) \Big]
$$
où $O(\frac{1}{n^3})$ représente les termes d'ordre supérieur ou égal à trois en $\frac{1}{n}$.

\question
Rappeler la définition de $n!$ où $n$ est un nombre entier positif.

\question
Calculer à l'aide d'une calculatrice de poche (ou de tout autre moyen dont vous disposez) $2!$, $8!$, $16!$ et $64!$.

\question
Jusqu'à quelle valeur de $n$ peut-on calculer $n!$ sur une calculatrice de poche standard ?

\question
Calculer de nouveau $2!$, $8!$, $16!$ et $64!$ en utilisant l'approximation suivante (dite d'ordre zéro) pour la factorielle
$$
n! \sim \sqrt{2 \pi n} \; n^n \; e^{-n} .
$$

\question
Calculer numériquement l'erreur relative $r(n)$ (c'est-à-dire le quotient $r(n)=\frac{n!-\sqrt{2 \pi n} \; n^n \; e^{-n}}{n!}$) pour $n=2, 8, 16$ et 64.

\question
Montrer que les résultats numériques sont compatibles avec $\displaystyle{r(n) \sim \frac{1}{12\, n}}$.

\question
On utilise la première expression de $n!$ pour calculer $\ln{(n!)}$. En déduire une expression de $\ln{(n!)}$ sous la forme d'une somme. 

\question
Jusqu'à quelle valeur de $n$ peut-on calculer $\ln{(n!)}$ sur une calculatrice de poche standard ?

\question
Calculer numériquement l'erreur relative sur $\ln{(n!)}$ faite en utilisant l'approximation d'ordre zéro pour $n!$ pour les valeurs suivantes de $n$ : 2, 8, 16, 64 , 1024, $10^{10}$ et le nombre d'Avogadro ${\cal N}_A$.

\question
Dans la question précédente,  quelle est la contribution du terme $\ln{( \sqrt{2 \pi n})}$ ? Est-il raisonnable pour $n$ grand d'approcher $ \ln{(n!)}$ par $n\,\ln{(n)} - n$ ? On pourra estimer à partir de quelle valeur de $n$ cette approximation est bonne à 0,1\% près.

\question
L'approximation $ \ln{(n!)} \sim n\,\ln{(n)} - n$ est connue par les physiciens sous le nom de \textit{formule de Stirling}. En prenant l'exponentielle de cette formule, peut-on affirmer qu'il est raisonnable d'approcher $n!$ par $ n^n \; e^{-n}$ ?
