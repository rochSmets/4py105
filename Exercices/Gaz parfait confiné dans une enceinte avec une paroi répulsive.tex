Un gaz parfait monoatomique de $N (\propto {\cal N}_A)$ particules sans interactions de masse $m$ est en équilibre avec un thermostat à la
température $T$ (on posera $\beta =1/(k_B T)$). Le gaz est confiné dans un récipient cubique de côtés $(0 \le x \le L,  0 \le y \le L, 0 \le z \le L)$ de volume $V = L^3$ dans un système de coordonnées cartésiennes $(0; x, y, z)$. Il existe une interaction répulsive entre la  paroi en $x = 0$ et chacune des particules du gaz. On appellera
$\vec{r}_i=(x_i,y_i,z_i)$ et $\vec{p}_i$ la position et la quantité de mouvement de la i-\`eme particule.


\question
Parmi les trois formes suivantes pour l’énergie potentielle $U(\vec{r})$ d’une particule localisée au point $\vec{r}=(x,y,z)$, quelle est celle qui correspond à une répulsion avec la paroi en $x = 0$ ? On justifiera sa réponse. Dans ces expressions, $\alpha$ est une constante positive.
\begin{center}
A : $U(\vec{r})=-\alpha x$; B : $U(\vec{r})=\frac{1}{2} \alpha x^2$ ; C : $U(\vec{r})=-\frac{\alpha}{x}$
\end{center}


\question
Écrire l'Hamiltonien ${\cal H}$ du gaz en fonction des quantités de mouvement $\vec{p}_i$, de $m$, de $\alpha$ et des abscisses $x_i$ des particules.


\question
Calculer la fonction de partition canonique $Z(T,V,N)$ du gaz. Montrer qu'elle s’écrit sous la forme
\begin{center}
    $Z =\frac{1}{N!}(\frac{V}{\Lambda^3})^N \left( \frac{\exp(\beta \epsilon)-1}{\beta \epsilon} \right)^N$ .
\end{center}
où $\epsilon$ ne dépend que de $\alpha$ et de  $L$, et $\Lambda$ est la longueur d'onde de De Broglie. 


\question
\'Evaluer $\Lambda$ à température ambiante pour du diazote. \`A quelle température $\Lambda$ atteint la taille caractéristique d'un atome ?


\question Montrer que dans la limite $\beta \epsilon \ll 1$, on peut approcher $Z$ par l'expression

\begin{center}
    $Z \approx \frac{1}{N!}(\frac{V}{\Lambda^3})^N \exp(N[\frac{\beta \epsilon}{2}+\frac{(\beta \epsilon)^2}{24}+\ldots])$.
\end{center}

On utilisera cette expression pour les deux questions qui suivent.

\question Calculer l’énergie interne $U$ ainsi que l’énergie libre $F$.

\question En déduire l’entropie $S(T,V,N)$. Déterminer $S- S_{gp}$, où $S_{gp}$ est l’entropie d’un gaz parfait de $N$ particules à la température $T$ dans un volume $V$. Interpréter le signe de $S- S_{gp}$.


\question Justifier que la probabilité $D(x) \D x$ de trouver une particule quelconque à une distance entre $x$ et $x+\D x$ de la paroi répulsive est donnée par l'expression $D(x) \D x = C\exp( \beta \epsilon \frac{x}{L}) \D x$. Que vaut la constante $C$ ?

\question En déduire la densité volumique de particules $n(x)$, puis la pression $p(x)$.

\question Tracer sommairement $p(x)$ et la comparer à la pression du même gaz $p_G$ en l'absence de paroi répulsive. On comparera notamment $p(0)$ et $p(L)$ à $p_G$.

\question Ce potentiel répulsif vous semble-t-il réalisable ? 
