On considère un système composé d'un grand nombre $N$ de molécules, dont les interactions sont négligeables, à l'équilibre avec un thermostat à la température $T$. Chacune des molécules possède quatre niveaux d'énergie \og interne \fg \, respectivement d'énergies -$\epsilon$, 0, 0 et  +$\epsilon$. On note $\beta=\frac{1}{k_B T}$. On rappelle les deux formules de trigonométrie hyperbolique : $\cosh(2u)=2 \cosh^2(u)-1$ et $\sinh(2u)=2 \sinh(u)\cosh(u)$.

\question
Quelle est la probabilité de trouver la molécule avec l'énergie -$\epsilon$ ? 0 ? +$\epsilon$ ? On pourra introduire $z(\beta)$ la fonction de partition d'une molécule.

\question
Montrer que l'énergie  moyenne $\overline{\epsilon}$ de la molécule est égale à $\overline{\epsilon}=- \epsilon \tanh ( \frac{\epsilon}{2k_BT})$. Quelle est l'énergie interne correspondante du système  ? Justifier votre réponse.

\question
En déduire la contribution $C$ de ces degrés de liberté énergétiques à la capacité calorifique \textbf{molaire} à volume constant de ce gaz.

\question
Tracer sommairement $C/R$ en fonction du paramètre $x=\frac{\epsilon}{2k_BT}$. Justifier pourquoi il y a forcément un maximum.

\question
Expliquer physiquement pourquoi ce maximum est obtenu pour $x \sim 1$. On pourra représenter sur un axe vertical les différents niveaux d'énergie et expliquer comment ils sont peuplés  dans les cas $k_BT \ll \epsilon$, $k_BT \sim \epsilon$ et $k_BT \gg \epsilon$.
