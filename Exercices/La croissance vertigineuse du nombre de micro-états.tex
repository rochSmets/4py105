On considère un système isolé de $N$ particules d'énergie interne $E$. On admet que le nombre de micro-états $\Phi(E,N)$  d'énergie inférieure à $E$ est de la forme $\Phi(E,N)=A(N)E^{\alpha N}$ où $\alpha$ est un nombre proche de l'unité.

\question
Rappeler la nature de l'énergie interne d'un gaz parfait monoatomique comme l'argon. Justifier sommairement pourquoi, dans ce cas, $\alpha=\frac{3}{2}$.

Pour la suite, on prend pour les applications numériques $\alpha=\frac{3}{2}, N={\cal N}_A$, le nombre d'Avogadro et $E= 6$ kJ.

\question
Calculer le ratio $\frac{\Phi(E+\delta E,N)}{\Phi(E,N)}$ pour $\delta E= 6 $ nJ puis  $\delta E= 0,375 $ eV.

\question
Expliquer pourquoi, pour toute valeur \og raisonnable \fg \ de $\delta E$, le nombre $\Omega_{\delta E}(E,N)$ d'états d'énergie compris entre $E$ et $E+\delta E$ près est égal à $\Phi(E+\delta E,N)$. Expliquer l'affirmation que \og dans les espaces de dimensions très élevées, tout le volume est dans la surface \fg.

\question
En déduire que, dans la limite thermodynamique, l'entropie $S_{\delta E}(E,N)=k_B \ln \Omega_{\delta E}(E,N)$ ne dépend pas de ${\delta E}$ et qu'on a
$$
S(E,N) \simeq k_B \ln \Omega_{\delta E}(E,N)\simeq k_B \ln \Phi(E+\delta E,N) \simeq k_B \ln \Phi(E,N). \nonumber \\
$$
