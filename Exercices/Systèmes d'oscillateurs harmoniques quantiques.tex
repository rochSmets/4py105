Un système constitué de $N$ oscillateurs harmoniques {\it quantiques} de pulsation $\omega$ à une dimension, discernables et indépendants, est isolé, son énergie étant égale à $E$. Pour rappel, l'énergie de l'oscillateur $i=1,\dots,N$ est donnée par $e_i=(n_i+\frac{1}{2})\hbar \omega$, où $n_i$ est le nombre quantique d'excitation de l'oscillateur.

\question
Donner l'expression de l'énergie du système en fonction des nombres quantiques d'excitation $n_i$, où $i=1\dots N$.

\medskip
D'après ce qui précède fixer l'énergie $E$ revient à fixer la valeur de la somme des nombres quantiques $n_i$ à une valeur
$M$ :
$$
\sum_{i=1}^N n_i=M
$$
Soit $\varOmega(N,E)$ le nombre de micro-états du système de $N$ oscillateurs dont l'énergie vaut $E$.
 
\question
Commencer par calculer $\varOmega(N,E)$ pour $N$ quelconque et pour $M=0,1$ puis 2.

\question
Calculer à présent $\varOmega(N,E)$ pour $N=2$ et $M=3$.
	
\question
Calculer $\varOmega(N,E)$ dans le cas général.\\
Indice : ce problème est équivalent à trouver le nombre de façons de répartir $M$ objets dans $N$ boîtes distinctes.

\question
Vérifier l'expression en calculant par une sommation directe la dégénérescence des niveaux d'énergie d'un oscillateur harmonique tridimensionnel ($N=3$), c'est-à-dire le nombre de possibilités d'avoir $n_x+n_y+n_z=M$.

\question
Dans la limite des hautes énergies ($N$ étant fixé et $M \gg N$), montrer que l'on a:
$$
\varOmega(N,E)\simeq {1\over (N-1)!}\left({E\over \hbar  \omega}\right)^{N-1}
$$

\question
En déduire l'entropie micro-canonique dans la limite des hautes énergies, puis la température. Inverser cette relation pour obtenir $E(T)$.
