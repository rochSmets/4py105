On rappelle que la densité de probabilité qu'une molécule de masse $m$
d'un système à l'équilibre à la température $T$ ait une vitesse $\vec v$ à ${\rm d} {\vec v}$ près est donnée, selon
Maxwell, par~:
\begin{equation*}
\text{P}(\vec v) = C \,{\rm e}^{-\beta \frac{m\vec v^2}{2}}\enspace,
\end{equation*}
où $\beta=\frac{1}{k_B T}$ et où $C$ est une constante.

\medskip

\question
Déterminer $C$ (la distribution de probabilité doit être normalisée).

\question
En déduire la densité de probabilité $F(v_x)$ que la projection selon l'axe $Ox$ du vecteur vitesse d'une molécule soit égale à $v_x$ à ${\rm d}v_x$ près.\label{theocine2}

\question
Calculer la vitesse moyenne $\langle \vec v\rangle$ d'une molécule.

\question
Calculer la vitesse quadratique moyenne $v_q$ d'une molécule, définie par $v_q^2=\langle {\vec v^2}\rangle$.

\question
Montrer que l'énergie cinétique de translation moyenne d'une molécule est $\langle e \rangle = \frac32 k_B T$.
