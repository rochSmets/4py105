Un gaz monoatomique et un solide cristallin, constitués des mêmes atomes sans spin de masse $m$ coexistent à l'équilibre dans une enceinte de volume $V$  à la température $T$. On néglige le volume du cristal par rapport à celui du gaz. La vapeur est assimilée à un gaz parfait classique. Le nombre total d’atomes est $N$, dont $N_g$ sont dans la phase gazeuse et $N_s=N-N_g$ en phase solide.

\question Rappeler l'expression de la fonction de partition canonique du gaz $Z_g(T, V, N_g)$, puis en déduire son énergie libre et sa pression. 

\begin{center} \vspace{-0.5cm}\line(1,0){0.2\textwidth} \vspace{-0.5cm}\end{center}

Dans le solide, les atomes sont situés aux noeuds discernables d’un réseau cristallin. Ils sont supposés indépendants
les uns des autres, le mouvement de chacun d’eux étant celui d’un oscillateur harmonique à 3 dimensions
(modèle d’Einstein). L'état d’un atome est caractérisé par 3 entiers positifs ou nuls $n_x$, $n_y$,
$n_z$ et son énergie est :
\begin{center}
$\epsilon_{{n_x,n_y,n_z}}=-\epsilon_0+\hbar \omega ( n_x+n_y+n_z+\frac{3}{2})$
\end{center}
où $\epsilon_0$ et $\omega$ sont des constantes caractéristiques du cristal.

\question Que représentent $\epsilon_0$ et $\omega$ ?

\question Exprimer la fonction de partition canonique du solide $Z_s(T, N_s)$. En déduire son énergie libre.

\begin{center} \vspace{-0.5cm}\line(1,0){0.2\textwidth} \vspace{-0.5cm}\end{center}

A priori, le nombre d’atomes sublimés $N_g$ peut prendre toutes les valeurs entre 0 et $N$. On se propose
de déterminer sa valeur à une température donnée.

\medskip

\question Comment s'écrit, en fonction de $Z_g$ et $Z_s$, la fonction de partition canonique de l'ensemble solide-gaz
en tenant compte de toutes les valeurs que $N_g$ peut prendre ?

\question Quelle est la probabilité pour que $N_g$ prenne une valeur particulière $N_g^0$ ?

\question En déduire la valeur la plus probable $N_g^E$ du nombre d'atomes dans la vapeur. Montrer qu'elle
correspond au cas où les potentiels chimiques des atomes dans le solide et dans le gaz sont égaux.

\question \`A température $T$ donnée, on s'intéresse aux variations de $N_g^E$ avec $V$. Montrer que pour $T$ et $N$ fixés, l'équilibre solide-vapeur n'est possible que si le volume $V$ de l’enceinte est inférieur à une certaine
valeur $V_M$ que l'on déterminera. Que se passe-t-il si on impose au système un volume $V$ supérieur à $V_M$ ?
