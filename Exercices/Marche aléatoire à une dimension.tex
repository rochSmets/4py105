Sur un réseau à une dimension, un marcheur ivre exécute une marche aléatoire de $N$ pas : à chaque pas de même longueur $l$, il va à droite avec une probabilité $p$ ou à gauche avec une probabilité $1-p$. On appelle $X$ sa position (relative à son point de départ) après $N$ pas.

\medskip

\question
Que valent la moyenne $\langle X \rangle$ et la variance ${\rm Var}(X)$ de $X$ ?

\question
Pour $p=\frac{1}{2}$, à quelle distance typique $d$ de son point de départ se trouve le marcheur après $N$ pas ? 

\question
Si la durée de chaque pas est égale à $\Delta t$, comment varie la distance $d$ avec la durée $t$ de la marche aléatoire ? Comment appelle-t-on ce type de comportement ?

\question
Comment peut-on écrire la distribution de probabilité de $X$ pour $N$ grand ?
