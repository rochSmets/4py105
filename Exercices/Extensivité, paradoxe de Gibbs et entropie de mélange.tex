Nous voulons clarifier le rôle de l’indiscernabilité dans le calcul de l’entropie du gaz parfait classique.

\question Rappeler ce qu'est la propriété d’extensivité pour une grandeur thermodynamique. Comment la vérifier sur l'entropie $S$ à partir de son expression $S(E,V,N)$ en fonction de l'énergie $E$, du volume $V$ et du nombre $N$ de particules du système. 

%\medskip

\begin{center} \vspace{-0.5cm}\line(1,0){0.2\textwidth} \vspace{-0.5cm}\end{center}

On rappelle que  le volume $\Phi$ dans l'espace des phases des états d'énergie inférieure ou égale à $E$ pour un ensemble de $N$ atomes de masse $m$ sans interactions confinés dans un volume $V$ en 3D est de la forme:
\begin{center}
$\Phi(E,V,N)=V^N \frac{(2\pi m E)^\frac{3N}{2}}{(\frac{3N}{2})!}$    
\end{center}

\question En déduire pour des atomes indiscernables le nombre $\phi$ de micro-états d'énergie inférieure ou égale à $E$, puis l'entropie microcanonique de ce gaz (formule de Sackur-Tetrode). Vérifier que l'expression est bien extensive.

\question Rappeler comment obtenir l'expression de la température $T$ et de la pression $P$ d'un système en fonction de son entropie microcanonique $S(E, V, N)$.
Exprimer $T$ et $P$ pour un gaz parfait monoatomique d'énergie $E$, de volume $V$ et à $N$ atomes.
\question En déduire une expression de l'entropie microcanonique uniquement en fonction de $T, P$ et $N$.

%\medskip

\begin{center} \vspace{-0.5cm}\line(1,0){0.2\textwidth} \vspace{-0.5cm}\end{center}

\`A la fin du XIXème siècle, il n’y avait pas de justification pour introduire dans le calcul précédent le facteur $1/N!$  nécessaire pour prendre en compte l’indiscernabilité des atomes. 

\question Donner l’expression de l’entropie microcanonique si les atomes étaient discernables (sans facteur $1/N!$).

\begin{center} \vspace{-0.5cm}\line(1,0){0.2\textwidth} \vspace{-0.5cm}\end{center}

\tiret{Paradoxe de Gibbs}

On considère deux gaz monoatomiques identiques (chacun d'énergie $E$, de volume $V$ et à $N$ atomes) occupant deux volumes égaux séparés par une paroi. 

\question Justifier que lorsqu’on enlève la paroi, l’entropie du système varie de $\Delta S_D = S(2E, 2V, 2N)-2S(E, V, N)$ et calculer cette variation. Pourquoi ce résultat est-il paradoxal ? Qu'aurait donné l'utilisation de la formule de Sackur-Tétrode ?

\begin{center} \vspace{-0.5cm}\line(1,0){0.2\textwidth} \vspace{-0.5cm}\end{center}

\tiret{Entropie de mélange}

On considère deux gaz monoatomiques non identiques, notés 1 et 2 (le gaz $i$ a une énergie $E_i$, un volume $V_i$ et $N_i$ atomes) occupant deux volumes séparés par une paroi mobile et diatherme.

\medskip

\question On considère qu'ils sont à l'équilibre mécanique et thermique. Que cela signifie-t-il ? On note $T$ la température et $P$ la pression dans chaque compartiment.

\question Calculer l'entropie de ce système avec la paroi en place en fonction de  $T$, $P$, $N_1$ et $N_2$.

\question On enlève la paroi. Que deviennent la température et la pression du système entier, mélange des deux gaz ?

\question Quelle est la pression partielle $P_1$ (respectivement $P_2$) du gaz 1 (resp. 2) ? En déduire l'entropie du système sans la paroi.

\question Calculer la variation d'entropie lorsque l'on a enlevé la paroi. Montrer qu'elle s'exprime comme

\mbox{$\Delta S = - k_B N \{ x \ln{x}+(1-x) \ln{(1-x)} \}$} où $N=N_1+N_2$ et $x=N_1/N$.

\question Pourquoi $\Delta S$ est ici non nul ?
