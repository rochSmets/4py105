On considère un gaz parfait constitué de $N (\sim {\cal N}_A)$ particules ultra-relativistes, {\bf indiscernables}, confinées dans un volume $V$ de dimension 3 : l'énergie cinétique $\epsilon_i$ d'une particule $i$ est donnée par $\epsilon_i= p_ic$ où $c$ est la vitesse de la lumière et $p_i=|\Vec{p_i}|$ la norme de sa quantité de mouvement. Soit $E$ l'énergie totale du gaz et $S(E,V,N)$ son entropie. On introduit $k_B$ la constante de Boltzmann et $h$ la constante de Planck.

\medskip

\question Comment relier $S(E,V,N)$ à $\Pi(E,V,N)$, {\bf volume} dans l'espace des phases des (micro-)états accessibles du gaz d'énergie inférieure ou égale à $E$ ? On commentera l'origine de cette expression.

\question Exprimer $\Pi$ sous forme intégrale. Montrer que $\Pi(E,V,N)=V^N (\frac{E}{c})^{3N} I(N)$ où $I(N)$ est une intégrale 3N-uple sans grandeurs physiques, dont on donnera l'expression.

\question On admet que $I(N)=\frac{(8 \pi)^N}{(3N)!}$. Le vérifier pour $N=1$. Calculer explicitement $S(E,V,N)$. On utilisera la formule de Stirling. Vérifier que $S(E,V,N)$ est bien extensive.

\question Calculer la température de ce gaz ainsi que sa pression à partir de l'expression de $S(E,V,N)$. Commenter.

\question On introduit $\Lambda=\frac{hc}{k_BT}$ la longueur d'onde thermique de de Broglie. Calculer $\Lambda$ à température ambiante. Exprimer $\frac{S}{N k_B}$ en fonction de $\Lambda$ et de $\rho=\frac{N}{V}$.  
