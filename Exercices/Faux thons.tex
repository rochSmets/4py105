Une cavité de volume $V$ et à l'équilibre à la température $T$  est le siège d'ondes électromagnétiques. On admet que ces ondes peuvent être décrites par une assemblée de photons obéissant à la statistique de Bose-Einstein à potentiel chimique nul. 

\`A une onde de fréquence $\nu$, on associe des photons d'énergie $\epsilon=h \nu= cp$ où $c$ est la vitesse de la lumière et $p$ l'impulsion (la quantité de mouvement) du photon.

\question
On admet que la densité d'états dans l'espace des impulsions est $g(p) d^3p =2 \frac{V}{h^3} 4 \pi p^2 dp$. (le facteur 2 est dû au fait que pour une impulsion donnée, un photon peut avoir deux états possibles de polarisation). Calculer la densité d'états en fonction de l'énergie.

\question
Quel est le nombre moyen   $n(\epsilon)$ de photons dans un mode d'énergie $\epsilon$ ?

\question
On désigne par $\rho(\nu, T)$ l'énergie par unité de volume des photons ayant leur fréquence comprise entre $\nu$ et $\nu+d\nu$ (appelée aussi densité spectrale d'énergie). Déterminer $\rho(\nu, T)$.

\question
La densité spectrale d'énergie est mesurable expérimentalement. Quelle est la forme de $\rho(\nu, T)$ aux basses fréquences (loi de Rayleigh-Jeans) ?  aux hautes fréquences (loi de Wien) ? Montrer que la loi de Rayleigh-Jeans peut se retrouver par le théorème d'équipartition de l'énergie. On admet que $\rho(\nu, T)$ présente entre ces deux régimes un maximum pour une fréquence $\nu_M$ qui vérifie $h \nu_M=2,82 k_B T$.

\question
Déduire de $\rho(\nu, T)$ l'expression de la densité d'énergie $u(T)$ dans l'enceinte. On donne
$$
\int_0^{+\infty} \frac{x^3 \ dx}{e^x-1}=\frac{\pi^4}{15}
$$

\medskip

On perce un petit trou de surface $A$ dans cette cavité d'où s'échappe une partie du rayonnement. On définit le pouvoir émissif $W$ de cette cavité comme étant le flux d'énergie sortant de l'orifice (par unité de temps et de surface).

\question
Calculer le nombre $d^3 n_{\epsilon, \Omega}$ de photons par unité de volume ayant leur énergie comprise entre $\epsilon$ et $\epsilon+ d\epsilon$ et leur impulsion pointant dans un angle solide $d\Omega$.

\question
En déduire le nombre de photons ayant leur énergie comprise entre $\epsilon$ et $\epsilon+ d\epsilon$ et leur impulsion pointant dans un angle solide $d\Omega$ qui quittent la cavité pendant le temps $dt$. 

\question
Calculer $W$ en fonction de $u(T)$.  Montrer que l'on a $W=\sigma T^4$ et calculer $\sigma$.

